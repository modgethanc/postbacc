\documentclass{article}[12pt]
\usepackage{mla}
\begin{document}
\begin{mla}{Vincent}{Zeng}{Hsu}{Trajectories in Photography}{\today}{Flakey Imagery}

Humans are visual creatures and scientists are ever hungry for more information. So, when Wilson Bentley, a quiet farm boy from Vermont, presented the world with photographic evidence of something that is now recognizable as the canonical representation of a snowflake, the world accepted his vision as the ideal that has always been desired. His images were the result of a deep personal obsession with what he believed to be the true, pure form of a snowflake, regardless of what actually fell from the skies. Despite that, the scientific community was willing to overlook flaws in his method for the sake of accessing data they previously did not have, and popular culture enthusiastically received the iconic image that filled a gap in the collective visual vocabulary.

The popular notion of a snowflake is a perfect, symmetrical figure---a flat, six-pointed shape that schoolchildren can cut from a carefully folded piece of paper. This iconic form is used as a universally recognizable symbol for snow, from winter-themed decorations to weather alerts that rely on the concision of pictograms rather than text. However, there was no widely accepted representation of snow prior to the publication of Bentley's photographic work in 1901. A spotty history of snowflake imagery turns up very few instances of recognition of that shape, being mostly drawings rendered from some combination of magnified viewing with the memory and imagination of the artist; snowflakes melted in a fraction of the time it took for a drawing to be made, and thus creating a faithful representation was both tedious and not really possible. 

At the age of fifteen, Bentley started his pursuit of the snowflake in 1880 with a microscope that was given to him as a birthday present. He took an approach similar to the snowflake documentarians of the past; he would catch snowflakes as they fell, look for one that seemed properly suited for illustration, then attempt to draw it freehand while observing the rapidly deteriorating form under a microscope (Blanchard 24). He found the process frustrating, especially as it highlighted the fleeting nature of his subjects, which he considered ``miracles of beauty; and it seemed a shame that this beauty should not be seen and appreciated by others'' (22). Here, though, he had already begun his process of manipulating the image of a snowflake; the sort of crystal he was attracted to was an extremely rare form, produced only under very specific weather conditions, and only through a lot of luck and patience was he able to capture one for his examination. As he was a hobbyist, not a scientist, his culling of ugly snowflakes could hardly be criticized. 

Once Bentley started taking photographs of his favorite snowflakes in 1885, his work took a different form; he was still only photographing snowflakes as a hobby, since his primary job was to work on his family farm and he could only entertain his need to document snowflakes as part of his leisure time. Still, he devoted as much time as he could every winter to chase his obsession, keeping detailed notes on weather conditions each day, as well as precise technical information about each exposure he captured (44--45). This level of detail and commitment to the task was a significant advantage to his entry into the world of scientists, even though he had no scientific background himself. Anyone who could manually produce hundreds of individually-captured snowflake photographs, as well as produce such an expansive amount of information about each one, easily gained the respect of the scientific community of his time.

A similar body of work produced just prior to his birth failed to gain serious recognition; an unnamed woman produced a book of drawings and poetry about snowflakes, and submitted them to a Harvard professor for his advice. Even as he respected the artistic merit of her work, he suggested that she present the drawings in a more scientific tone, ``otherwise the whole would be useless to students of nature, and only to be looked at as an elegant toy, fit to excite the curiosity, but not to impart information'' (61--62). Her work was apparently not well popularized, perhaps as a combination of not reaching the right audience at the right time, and not having enough appeal as either art or science.

Bentley's work, on the other hand, saw almost instant success once it fell into the hands of meteorologists. Beginning in 1899, he started selling his snowflake photographs to scientists at five cents per print, including his detailed technical notes with each image. The prints were made from duplicate negatives, which he carefully etched in an attempt to improve the appearance of the details of the flake; because it was virtually impossible to capture an entirely clean image of a perfectly symmetrical, unbroken snowflake, he allowed himself to modify the photographs that were ultimately to be shared, just to make sure that his vision of the pure snowflake was sustained. The resulting photographs were so impressive that his modifications were generally overlooked in favor of being able to study the overall body of work.

At this stage, he was very open and forthcoming about his methods, including notes that his photographs neglected to show an average view of snowflakes. In a letter to meteorologist Abbot Rotch included with an order of photographs, he readily admitted that ``the average forms are much less perfect \& beautiful'' (68), and described what sorts of weather conditions produce the more perfect snowflake he sought. Despite this open declaration of his selectivity in providing specimens for scientific study, his images were rapidly acquired by the scientific community. Rotch's response did not even comment on Bentley's obvious bias for aesthetic beauty; he only encouraged Bentley by paying for extra photographs and spreading the images to other people who were studying weather phenomena (68).

By 1902, Bentley became known for his skill in photographing snowflakes, a task that had not been performed so successfully due to the technical challenges with the equipment available at the time. As a result, his work was in demand by multiple publications and studies; on request, he produced a ten page article with 255 photographs of snowflakes for the \textit{Monthly Weather Review}, a scientific journal of meteorology, and thus solidly set himself a place at the scientists' table, despite still being a potato farmer from Vermont. For that article, he understood the need to document a more average selection of snowflakes; despite that, he had a hard time shaking his desire to show only the snowflakes he found worthy. He admitted: \begin{quote}	This proved the most difficult task of all, because the old habit of seeking for the beautiful and interesting, rather than the characteristic types, was very difficult to overcome. For this reason, I fear the winter's photographic record portrays far more fully the general character of the beautiful and interesting than it does the broken or unsymmetrical types. And yet there are few, perhaps, who after viewing the feast of beauty filling these pages will regret our shortcomings in this regard (88).\end{quote} In the current standards of scientific publication, it seems implausible that such obviously aeshetically-driven work produced by someone with no training in scientific process should be published as legitimate research.

Bentley was producing his photographs at a time when the role of photography was not yet well defined in any field---artists looked at photography with great skepticism, scientists had yet to understand its potential utility, and no real commercial venture existed. In that respect, the fact that Bentley's background did not have any obvious ties to the current modes of photography might not even have mattered. At the time, no one knew how to categorize the act of photography; all that was relevant was that someone had managed to cheaply mass-produce images of something that previously hadn't been easily seen with such clarity, and for the sake of scientific and cultural progress, that work was accepted regardless of its haphazard provenance. In other words, he filled a need that society had, during a time when both images and information were insatiably consumed in a post Industrial Revolution era.

There's a relationship that exists between the photographic process and the notion of objectivity. An attractive belief is that the camera's inability to create a falsehood and photography's reliance on the visual existence of the actual object being photographed implies an indisputable source of objectivity. In Bentley's case, he attempted to capture an objective representation of a subjective beauty---that being snowflakes in a state of purity, untainted by their existence in the real world. Rather than document the specimens he considered faulty and unworthy, he distilled his vision by not even allowing those natural faults to make it onto film at all. However, the reality of snow does not lie in what he decided to show; it lies in a more average, realistic selection of snowflakes that never appeared in his portrait of snow. Furthermore, the images that he chose to share were individually manipulated to display his belief of each snowflake's ideal beauty; even though he made unmodified, original negatives available for inspection, the point is moot because only modified images ever entered the popular view.

Regardless of what real snow actually looks like, the process of making a recognizable illustrative image necessitates an abstraction of the original form into a simpler, more appealing shape. Ideography is just that---turning ideas into simple pictures. The picture only needs to bear a slight resemblance to the original, and also needs enough pervasion in popular use to become accepted as a canonical graphic. Manipulated or not, Bentley's photographs provided a crisp graphic form that was easily mimicked in illustrations, and also experienced the pervasion through cultural media that allowed the form to become popularized. Art does not require objectivity in order to be successful; Bentley's photographs as just works of art were accepted because he was able to show objects of beauty in a way that were never seen before. Conveniently, his photographs showed snowflakes in a way that made for an obvious ideograph.

Bentley's work was only successful because it came during a period of shifting standards of objectivity. In current scientific communities, his work wouldn't stand a chance at gaining recognition as a rigorous documentation of a natural phenomenon due to his selection bias and manipulation of captured images. However, at the time he was producing his work, such a body of images did not yet exist, and there were no facilities capable of creating them. The fact that there was such a sheer volume of images that no one had seen before, coupled with his detailed technical notes and willingness to practically give the work away, created exactly the sort of situation for his work to become popular. His obsession with the pursuit of cataloguing his own notion of beauty and perfection resulted in the canonical snowflake image today.

\begin{workscited}
	
	\bibindent 
	Blanchard, Duncan. \textit{The Snowflake Man}. 1998.
	
\end{workscited}
\end{mla}
\end{document}
