\documentclass{article}[12pt]
\usepackage{mla}
\begin{document}
\begin{mla}{Vincent}{Zeng}{Hsu}{Trajectories in Photography}{\today}{Analysis of Wilson Hicks: ``What Is Photojournalism?''}
	
	Hicks occupied a position that gave him a substantial amount of authority on photojournalism; he contributed to the medium in a way that made him a witness to the evolution of photojournalism, just as the photojournalist was a witness to news stories. That said, it also put him in a position of power as a firsthand reporter on something he helped to create. Overall, he writes in concrete, absolute statements, often presenting questions or counter-arguments immediately after his point and then addressing them---an apt reflection of his position as the journalist of photojournalism.
	
	To address the question he posed in his title, Hicks breaks the problem down into two approaches: first, he defines photojournalism from within by excising its components and asserting how each segment functions as part of the whole medium; second, he describes photojournalism externally by providing an overview of the social and technological developments worldwide that needed to occur in order for photojournalism to have a place. Both approaches are necessary for a complete statement on an otherwise complicated medium with many variables and expectations; the initial technical definition comes first as a foothold for the reader to understand the social structure that comes second.
	
	By Hicks's technical definition, photojournalism is a combination of words and pictures; more directly, it is one unit of expression that affects both the eyes and the ears of the reader (words, he states, as a means to stimulate the ears, and pictures to stimulate the eyes). This unit combines with what he calls the `X-factor'---the ideas, memories, and beliefs carried by the reader---in order to form an experience that is greater than what each of the three segments could have provided alone. With this, then, the existence of an audience is crucial to the success of Hicks's photojournalism. In the production of photojournalism, he ascribes each piece to a separate role: the writer, the photographer, and the editor, that even if one person holds multiple roles, all three must exist in a balanced, cohesive group in order to achieve success.
	
	Additional to the three segments of a unit of photojournalism, Hicks deconstructs how the photograph itself contributes to the experience. He points out that in reality, human vision is limited, affected by the mind's capacity to observe and process imagery; the camera, in that sense, has a superior perception of the visual world, able to mechanically record images without the burden of emotions. Furthermore, photographs gave people the ability to examine an image free of the immediacy of reality, and thus allows the formation of a new sort of emotional response not previously possible. It's this particular power of photography that Hicks claims as a major contributor to the success of photojournalism.
		
	In tracing the social trajectory of photojournalism, Hicks attributes the turning point to the technological advancements offered by the Leica, a camera that was so small and unobtrusive compared to previous machines. Previously, the presence of a camera unavoidably changed events due to the physical intrusion of a camera and all its related entourage, but once the actual act of photography was simplified, entirely different photographs were possible---rapid sequences, low light, and less arranged shoots, to name a few. Coupled with the ability to distribute photographs in print media, photographs became a more integrated part of journalism, rather than just an illustrative or design element to accompany text. With the Leica, photographers were able to speak to the point that Hicks makes about why photography is effective: the non-intrusive images showed scenes that viewers could interpret as if they were there, but without the problems of faulty perception and memory. Additionally, photographers themselves could hone and rely on their instincts for capturing images with the potential to have this effect.
	
	The interesting part about Hicks's account of the development of photojournalism, specifically with regards to \textit{Life}, is that his syntax does not give away, or even imply, that he was involved heavily with its production. However, the breadth of detail and absolute descriptions give a sense of witness to that development; it's as if he wanted to see photojournalism in the way that the camera saw events, with the ability to record and not judge. That said, the selectivity of the camera applied to Hicks's position as well; the more absolute his statements, the narrower a path he must take, and his perspective did not allow intrusions beyond the definitions he carefully laid out.
	
\end{mla}
\end{document}
