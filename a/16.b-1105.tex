\documentclass[12pt]{article}
\usepackage{mla}
\begin{document}
\begin{mla}{Vincent}{Zeng}{Professor Paul Eiss}{Art, Anthropology, and
  Empire}{\today}{Commodifying Tattoo}

Samoan tattooing is a cultural practice that encompasses complicated facets of
social development, colonization, and artistic exchange. In modern settings, it
can serve as a visible and ineffable reminder of ethnic heritage and social
standing, but it has also been commodified as artistic labor for consumption by
tourists. As visual culture and globalization increased the exchange of ideas
between \textit{tufuga} and non-Samoan tattooists, issues developed around
ownership and license to the motifs, rituals, and tools associated with the
practice.

The Samoan tattooing process is called \textit{tatatau}, from ta (`to strike')
and \textit{tatau} (images drawn on the body)\footnote{Mallon, p. 145}.
Specifically, Samoan men receive \textit{tatau} in the form of long, parallel
lines, dark shading, and detailed geometric shapes that extend from their hips
to their knees; this general patterning is called \textit{pe'a}\footnote{ibid.},
and men often complete their \textit{tatau} over many sessions throughout their
life. The time, expense, and pain it takes to complete a \textit{pe'a} serves
both the practical reason of needing to let the skin fully heal between
sessions, and as a cultural marker to represent growth, coming of age, and
maturity.

\textit{tatau} are created by a \textit{tufuga tatatau}, a skilled practitioner
who is often associated with a traditional guild or family where the techniques,
tools, and specific motifs are passed down through generations. Traditional
tools are made from a variety of small bone combs that produce different types
of lines, which are attached to a shell plate that binds it to a wooden handle.
This handle is then struck repeatedly with a rod to drive the fine teeth of the
bone comb into the surface of the skin, which deposits the pigment into place.
Each \textit{tufuga} has the license to create a \textit{tatau} with variation
in specific lines and motifs, depending on the shape of the body and other
imagery appropriate for a particular person; still, the overall appearance and
style maintains a persistent, identifiable look, which gives the impression of a
longstanding ancient practice. Contemporary practitioners, both of Samoan and
non-Samoan origin, are known to incorporate \textit{tatau} imagery in their
work, but largely do so with modern electric machines\footnote{ibid., 146}.

Traditionally, a \textit{tufuga} would accept customary Samoan exchange goods
such as \textit{'ie toga} (fine mats), \textit{siapo} (bark cloth), and pigs; in
a modern setting, in which access to such goods has decreased (especially in the
diaspora of Samoans living outside of Samoa), cash payments starting around
\$1800 have been accepted substitutions\footnote{ibid., 159}. However, the
introduction of cash into the \textit{tatatau} process inevitably raises
questions of commercialization, especially as it coincides with the rise of
tourism and commodification of indigenous art.

One of the earliest Samoan \textit{tufuga} to practice prominently outside of
Samoa was Su'a Sulu'ape Paulo II, who worked in New Zealand after 1972, along
with his brothers, who also worked in the United States and Europe. In addition
to bringing the \textit{pe'a} to the Samoan diaspora, Sulu'ape Paolo also
traveled to tattoo conventions abroad and tattooed non-Samoans, including the
highly publicized process of tattooing the artist Tony Fomison\footnote{ibid.,
157}.  Non-Samoans were known to receive traditional Samoan tatau; European
explorers and traders sometimes collected \textit{pe'a} on their travels through
the 1800s, and Tongan elites have traveled to Samoa specifically to receive
tattoos\footnote{ibid., 149}; however, the level of exchange in which Sulu'ape
Paulo engaged with outsiders was a controversial topic for more traditional
Samoans. Sean Mallon explains that ``[when] non-Samoans are tattooed, a custom
Samoans feel they have some authority over slips beyond their influence and
control.''\footnote{ibid., 158}

This controversy seems to reflect on a larger cultural response to the
appropriations that Nicholas Thomas describes; he mentions ``[museums] crammed
with indigenous artifacts''\footnote{Thomas, 125} as collections that represent
the European collector's ``curiosity'' in these items. This curiosity stood as a
precursor to colonialism, as it was often a response to practices deemed
primitive or barbaric, and a euphemism that avoided ``passing an aesthetic
judgment''\footnote{ibid., 129–130}. By collecting these artifacts for display
in natural history museums, those items lost the meanings of their original
contexts, and ultimately supported European attitudes towards colonizing and
civilizing people perceived to be inferior.

The collecting and commodification of \textit{tatau} is more complicated;
creating one involves a tedious and painful process, which results in a design
that exists on a human body. The very fact that the images are embedded in
living flesh is how Sulu'ape Paolo justifies allowing non-Samoans to receive his
work; Mallon reports that ``[he] considered the \textit{tatau} something the
wearer could not exchange or sell''\footnote{Mallon, 158}, which effectively
renders the tattoo longer a commodity in and of itself. One cannot simply remove
a tattoo from someone's body for display in the sort of museums that Thomas
described, which protects it from that specific method of decontextualization.

However, the completed \textit{tatau} itself is not necessarily what has become
commodified, so much as the labor of the \textit{tufuga} and access to the
visual motifs. In the documentary Cannibal Tours, we can clearly see tourists
traveling to foreign, ``primitive'' places specifically for the purpose of
acquiring souvenirs, which included photographs of themselves with indigenous
people receiving traditional face paint. That the natives of those areas express
frustration and confusion at this behavior shows that they feel something is
being taken from them, even if the loss is not tangible; it's the idea of having
visited that place, made connections, and acquired proof that motivates tourists
to collect experiences of other cultures. Mallon notices this trend in the form
of postcards, pamphlets, and Internet forums featuring tattooed Samoans, which
generate outsider interest in Samoan \textit{tatau}.

A consequence of visual accessibility to the \textit{pe'a} is that a non-Samoan
with tattooing equipment could reproduce those motifs without having any
association with the traditional practice. A tattooist in Pittsburgh who worked
on one of my tattoos said he often works with unfamiliar designs that his
clients bring in themselves, and generally doesn't ask about their origins
because he believed it was the clients' responsibility to approve or reject what
imagery they wanted to wear. Meanwhile, a friend of mine who works in Denver
withholds her labor from client inquiries that she deems as cultural
appropriation (to which she includes mimicry of indigenous styles) because she
does not want to knowingly contribute to the commodification of other cultures.
The prevalence of popular ``tribal'' style tattoos on Westerners, as well as the
range of ethical decisions made by modern tattooists, indicate ongoing issues
surrounding the rights to motifs in indigenous designs.

The work of both traditional \textit{tufuga} and modern tattooists can
perpetuate the appropriation of Samoan \textit{tatau} motifs, regardless of
restrictions upheld by certain parts of the culture. However, \textit{tatau}
does not exist as a static, ancient practice; its usage has evolved continuously
as a response to encounters between Samoans and non-Samoans. The exchange of
imagery, economics, and nationalities inevitably changes the context of a
practice that is central to a cultural identity.

\begin{workscited}
  \bibent
  Mallon, Sean. ``Samoan \textit{Tatau} as Global Practice'', from
  \textit{Tattoo: Bodies, Art, and Exchange in the Pacific and the West}.
  Durham, 2005. pp 145--169

  \bibent
  Thomas, Nicholas. ``The European Appropriation of Indigenous Things'', from
  \textit{Entangled Objectts: Exchange, Material Culture, and Colonialism in the
  Pacific}. Harvard, 1991. pp 125--84.

\end{workscited}
\end{mla}
\end{document}
