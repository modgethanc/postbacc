\documentclass[12pt]{article}
\usepackage{mla}
\begin{document}
\begin{mla}{Vincent}{Zeng}{Professor Paul Eiss}{Art, Anthropology, and
    Empire}{\today}{Recontextualizing Aesthetics of a \textit{Nkisi Nkondi}}

The Carnegie Museum of Art in Pittsburgh includes a room designated as ``African
Art''. In it, visually arresting wooden sculptures sit on pedestals and
inside display cases; a headdress from Guinea faces a house post from Yoruba,
surrounded by stoic masks and aggressive human figures at varying altitudes. One gets
the sense that there are staring faces everywhere, blank expressions peering out
of glossy wood and frozen gestures. It's a view consistent with the European
impression of Africa as a mysterious continent of dark faces and unwelcoming
figures. We're captured by this sight; this is an art museum, after all, and the
aim of artwork is to move the viewer in some way.

In this room, the \textit{nkisi nkondi} on display bears a placard identifying it as such,
translating the native term to ``nail figure'', and names its origin as
``Congo'', between 1880--1920. As is standard for describing artwork, the placard also lists the
materials used: ``wood, pigment, iron, ivory, cotton, and other materials''. The
figure itself stands approximately three feet tall, but on a chest-high pedestal
so that the top of its clenched fist comes far above the head of anyone standing
before it. With its open mouth and wide, bright eyes, the \textit{nkisi} appears
in mid-attack, one foot stepping forward towards the front of its glass case.
However, its motion is arrested by hundreds of nails---the `iron' mentioned on the
placard---driven into its torso, feet, and neck, ringing its skull and face. Some of the
nails secure chains; others are tied with bits of cloth or seed pods or shells
(`cotton, and other materials', thanks to the placard). Do we feel safer seeing
the figure contained behind glass, or do its adornments transfer to us sympathy
pains of having nails driven into a body? In any case, the figure, physically
elevated and lit to cast jagged shadows onto the ground, calls on our empathy
for the recognizably humanoid shape.

Such is the context that this particular \textit{nkisi} creates. Because it
appears in a traditional western art museum, viewers are cued to respond to it
in an emotional, aesthetic framework. However, the display lacks things that artwork
in a neighboring room provides; there's no attribution to the artist that created
the piece, nor a statement of intent. In fact, nothing in the African Art room
has this information.

Examining the process of acquiring these pieces addresses some of these gaps of
information. Wyatt MacGaffey identifies most Kongo objects on display as part of
collecting efforts by European colonizers of that region between 1880 and
1920 (the same time frame indicated by the placard for the Carnegie's
\textit{nkisi}). Because colonizing is an act of domination, these visually
arresting objects were taken to show evidence of the primitive nature of the
natives. In that sense, individuals who created those pieces were not deemed
important, nor was there any need for the collectors to understand the intended
usage\footnote{MacGaffey, 172}. The Carnegie's \textit{nkisi} includes a note
that it was a gift of the Carnegie Museum of Natural History; we can infer,
then, that this transfer signified a re-contextualization of the figure from an
artifact of a past culture to a current representation of aesthetics. This
process brings such an object into its position of display in a western
art museum.

To fully contextualize this \textit{nkisi}, we then need to understand its
original setting. For the Kongo people, a \textit{nkisi} (plural.
\textit{minkisi}) is a general category of vessels that contain some sort of
medicine\footnote{ibid, 175}, often created by a \textit{nganga}, who was a
community figure with access to knowledge of both spiritual and physical
healing\footnote{ibid, 173}. An individual \textit{nkisi} can provide any of a
multitude of uses, such as warding against witchcraft, capturing thieves, or
curing ailments; the forms that the figures took would represent the usage,
either literally or symbolically. MacGaffey attributes the variety of
appearances to these different uses, along with the important fact that ``if one
was thought not to work it would be discarded and a new one
created''\footnote{ibid, 176}. This placed the \textit{minkisi} in their
original context as objects of use and function, before some of them became
figures of aesthetic interest for western museum curators.

The \textit{nkisi nkondi} in particular refers to a type of \textit{nkisi}
created to maintain social order (the word \textit{nkondi} meaning
``hunter''\footnote{Appiah, 260}). These \textit{minkisi} took on the shapes of
aggressive humanoids that embodied a spirit or a powerful force that could hunt
down a thief or an oath-breaker. As part of the ritual usage, one would drive
into it nails or blades that represented the problem one wanted solved, or swore
oaths in front of it, sealing them with a token of adornment added to it. Appiah
aptly points out that ``[t]he extraordinary visual impact of these figures,
covered, porcupine-like, with these jagged metal extrusions, is thus the result
of the process of using them.''\footnote{ibid, 260}. The aggressive, fearsome
look of a \textit{nkisi nkondi} served as a potent visible reminder of law and
order, a physical representation of the invisible forces that allowed a community
to function. Over time, the figures collected additions, growing in both its
visual appearance and its spiritual power, and containing values that the
colonizing forces ignored when they removed the \textit{minkisi} from that
setting.

As the Carnegie's \textit{nkisi} left its original context, it lost
its power as an object of ritual use and spiritual function. MacGaffey states
that ``[a] \textit{nkisi} in a museum is inert and in fact dead'', owing to,
among other reasons, the fact that the \textit{nganga} that created it is
no longer alive\footnote{MacGaffey, 176}. Holes in that \textit{nkisi} indicate
places where original nails might have fallen out over time, but we cannot
forget that the hairs wrapped around some of the remaining nails once came from
a living creature, driven into the wooden body to represent a goal we do not
know. By its display as a work of art rather than utility, it then gains power
that is purely aesthetic, causing viewers to respond emotionally with no
knowledge of its original intent.

The \textit{nkisi nkondi} as seen in the Carnegie shows more than just one
act of an individual's artistic insight; it's an object that represents the
collective ambitions of an entire community. Originally a tool of ritual group
use, it nonetheless gained aesthetic value by its nature of invoking an
emotional response, both through the traditional process of physically adorning
it with a symbol of a particular desire, and through its current status as an
imposing figure displayed in a room of other similarly imposing figures. The
placard associated with it gives viewers the barest sense of what it is, but
ultimately leaves the task of understanding the \textit{nkisi} up to the
response one has to its aggressive, demanding appearance.
\begin{workscited}
  \bibent
  Appiah, Kwame Anthony. ``The Arts of Africa'' in English, Richard and Skelly, Joseph, \textit{Ideas Matter: Essays in Honor of Conor Cruise O'Brien}, 251--264

  \bibent
  MacGaffey, Wyatt. ``Meaning and Aesthetics in Kongo Art'' in Cooksey, Susan, \textit{Kongo Across the Waters}, 172--179.

\end{workscited}
\end{mla}
\end{document}
