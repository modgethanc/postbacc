\documentclass[12 pt]{article}
\usepackage{mla}
\begin{document}
\begin{mla}{Vincent}{Zeng}{Sandage}{AMRN PTLT HMR MK TWN}{\today}{The Importance of Being Obnoxious}
The self-publishing platform, lack of regulation, and relatively anonymous means of content distribution provided by the internet allows for certain types of humor to develop that could not flourish under any other circumstance. In particular, the nature of the internet promotes a sort of derisive humor, as it lowers the social risks normally present in making and enjoying distasteful jokes or deliberately offensive commentary. \textbf{Through the collection of articles on \textit{The Best Page in the Universe}, pseudonymous humorist Maddox demonstrated that on the internet, sometimes it's effective to be an asshole.}

	\textit{The Best Page in the Universe} started in 1998, and has remained stylistically unchanged from its original format of bland white text on a black background with no other features other than the occasional colored text for emphasis and badly drawn graphics to illustrate particularly rude points. The only perceptible modernization is the addition of buttons at the bottom of each article that allow the viewer to link that particular page with one's social network of choice. His website followed a common path of websites; it started as a personal space for rants that spread by word of mouth until he reached the status of internet superhero for his unapologeticallly scathing essays about punching women, killing babies, and the depressing stupidity of mankind. Curiously enough, and perhaps indicitive of the egotism and self-awareness frequently seen in high-profile comedians, Maddox pointed out his own fame in the aptly-titled article ``How is it possible that a guy with a small penis and a hairy back is more powerful than Pepsi on the Internet?'' in which he charted money spent on advertising versus Alexa.com page ranking. Although he never offered an answer to that question, he painted the incongruous image that ``corporate sites have promotions, games, and discounts, yet more people come to this site every day and read stupid bullshit about how big my balls are'' (Maddox).

	Maddox's questioning of his own position on the internet highlights the recurring theme seen in a history of political humorists, that ``[t]he humor arises out of the gap between the cultural ideal and the everyday fact, with the ideal shown to be somewhat hollow and hypocritical, and the fact crude and disgusting'' (Rubin, 262). In fact, the internet itself may represent the final frontier of Rubin's Great American Joke; it presents a platform where anyone with a computer and a connection can place content in the same space, which often places eloquently worded commentary next to incomprehensibly illiterate babble. Even if a particular website only contains polished, edited, and serious material, there's no stopping a user from having that page open next to pictures of cats that look like Hitler. It is, as Rubin says, ``a society based theoretically upon the equality of all men...and the incongruities are likely to be especially observable'' (Rubin, 263). In such a world, the fact that Maddox's crude and frivolous articles display a measurably stronger hold on viewers than carefully crafted online brands show that there must be something viewers want that he is providing.

	In \textit{Cracking Up}, Paul Lewis points to ``an eagerness to provoke and be provoked'' as the basis for derisive humor, that jokes have evolved to having a specific target and agenda in mind (Lewis, 6-7). Once this crosses into the realm of aggressive and violent jokes, he asserts that it ``bars the butt/victim from joining in the laughter and puts the viewer in the awkward  position of laughing with a monster, refusing to do so, or sustaining an uneasy ambivalence'' (Lewis, 25). An interesting twist on Lewis's butt wars is that Maddox explicitly targets his readers in his attacks. Following an April Fool's prank in which he drastically altered the format of his website, he posted ``How do you dumbasses  manage to breathe?'' as a response to the flood of bewildered, concerned, and sometimes outraged responses to the appearance of his website. He ends the article with ``[a]fter reading a few thousand emails like the ones above, I seriously considered taking down my site and just posting links to animal porn for you retards. You're all idiots, and I've lost what little respect I've had for you'' (Maddox). Despite being constantly openly hostile and degrading to his readers, people keep coming back, as evidenced by the page view counter on that article; it reads over two and a half million visitors eight years after it was originally published.

	Perhaps what brings readers back, though, is knowing that they personally were not the ones writing Maddox stupid emails. Despite the fact that Maddox directly addresses the reader and unabashedly generalizes all of his readers to the same label of ignorant twats, any particular reader can enjoy the jokes by knowing that he only really meant people stupid enough to send him stupid emails. Like Lenny Bruce reading court transcripts as part of his stand-up act, Maddox publishes choice examples of hate mail, offering both dismissive commentary and scathing rebuttals. While other writers on the internet will leave open the option to comment on articles directly, Maddox maintains the old-fashioned method of publishing letters-to-the-editor as he sees fit; his audience has no way to verify the authenticity of any submission, and those emails become further props and objects of ridicule. 

	Despite the generally immature nature of his articles, Maddox often touches on political topics with serious social consequences. At the heart of his articles is an impressive ability to get extremely angry at virtually anything, and he treats political issues that strike him as unforgivingly stupid just as badly as he treats his readers who can't tell when he's pulling an April Fool's joke. Corruption of the police, hypocrisy of activists, partisan hackery, and censorship are frequently recurring topics, particularly as they highlight the incongruities at the base of every Great American Joke.

	In addition to just ridiculing politics, though, he has leveraged his readership for calls to action. During the January 2012 web-protests against the Stop Online Piracy Act, he wrote ``I hope SOPA passes,'' an article that, contrary to its title, didn't promote the highly controversial bill, but instead questioned the effectiveness of protests that only raise awareness of issues without making attempts to effect change. His refusal to participate in the protest by blacking out his website for a day was ``because it doesn't address any problems, only the symptom'' (Maddox), referring to his belief that even defeating one particular bill wouldn't make a difference because lobbyists would just push a new one. Not willing to mark himself a hypocrite for describing problems without offering solutions, he compiled a list of companies known to support the bill, encouraging his readers to boycott them, and additionally opening up discussion for boycotting tactics and other effective ways to fight the greater problem at hand. He asserted that defeating SOPA was ``like trying to stop a cold by blowing your nose. It's time we go after the virus'' (Maddox). 

	Peter Robinson explored an ongoing relationship between comedians and politicians through their influences on the public; he concluded that ``the political comedians would be invested with the sort of comprehensive sovereignty that presidents and candidates can only dream of: political influence, economic power, and cultural celebrity'' (Robinson, 214). In Maddox's case, his status as a celebrity comes first, and his cult following generates the rest. He directs a naturally abrasive persona towards producing humor that is, on the surface, degrading and offensive to virtually everybody, but he wouldn't have the popularity that he does if there wasn't a current of truth running through his words. His essays are satirical, gross exaggerations of problems in society that people are already aware of, only he challenges how people think about them in addition to pointing out that they exist.

\begin{workscited}

\bibent Lewis, Paul. \textit{Cracking Up: American Humor in a Time of Conflict}. 2006.

\bibent Maddox. ``I hope SOPA passes''. \textit{The Best Page in the Universe}. 30 January 2012. 

\bibent Maddox. ``How do you dumbasses manage to breathe?''. \textit{The Best Page in the Universe}. 05 May 2004. 

\bibent Maddox. ``How is it possible that a guy with a small penis and a hairy back is more powerful than Pepsi on the Internet?''. \textit{The Best Page in the Universe}. 18 August 2003.

\bibent Robinson, Peter. \textit{The Dance of the Comedians}. 2010.

\bibent Rubin, Jr., Louis D. ``The Great American Joke''. 1973.
\end{workscited}
\end{mla}
\end{document}
