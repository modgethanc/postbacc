\section{Misunderstanding the Misunderstood}

\textit{9 October 2014}

Through his living experiment detailed in \textit{Walden}, Henry David Thoreau
calls into question the cultural expectations of a capitalist work environment.
In spite of the fact that he spent that period of his life living physically and
socially separated from the community, he placed himself in a position to
uniquely observe and cogitate on the pressures to acquire capital.
\textit{Walden} provides a viewpoint that challenges ideals about work held by
the people who frustrated Thoreau, and why they participate in a society that
perpetuates that particular culture. He wrote from the only perspective that
such observations were possible, and that was a perspective that made it equally
hard for people to understand him.

While describing his day-to-day life at Walden Pond, Thoreau constantly
demonstrates his reliance on the existence of a community beyond himself, such
as borrowing an axe to build his house,\footnote{Walden, 36} buying shoes from
the town store\footnote{ibid., 153}, even seeking shelter at a neighbor's house
in a storm\footnote{ibid., 181}. In doing so, he communicates that he is not in
fact just some hermit in the woods, but aware of commerce and industry outside
of his personal bubble and willing to interact with such. Yet, he also claims
that he spent his Walden time living only by his own labor\footnote{ibid., 4,
61}. This is a paradox that comes off as hypocritical, and a confusing front for
the reader to process.

Thus, \textit{Walden} provides a viewpoint that is difficult to discuss or
reconcile. Thoreau comes right off the bat saying that readers of his
book are poor, ignorant, dishonest\footnote{ibid., 6}, and that not all of
Walden will be relevant to all people\footnote{Walden, p. 4}. By doing this,
he's setting himself up to be misunderstood from the very beginning---a fact of
which he is clearly aware, and often reiterates. He even gives an entire case
study of meeting someone who is not at all equipped to take on his
lifestyle\footnote{ibid,  ``Baker Farm''}, as if the reader required an explicit
illustration of how at-odds with the status quo Thoreau's life presents.

Despite detailing how he has avoided a majority of gainful employment, Thoreau
does not order the reader to reject work, but to work carefully, in a way that
does not damage others. He is convinced of the ability of everyone to provide
for themselves in a way that uses less of everything---money, energy, food,
etc.---and that the economic system of capitalism leans on the fact that
people don't, and instead are wasteful and have mindless approaches to work.
In essence, this is a direct response to what the sociologist Max Weber
later defined as the spirit of capitalism. In reference to the mindset
required to sustain a capitalistic system, Weber asserts that ``[b]ecause
these ethical qualities were of a specifically different type from those
adequate to the economic traditionalism of the past, they could not be
reconciled with the comfortable enjoyment of life''\footnote{Weber, 30}.
Thoreau believed that the capitalist work ethic damaged some core sense of the
human condition.

``I have made some sacrifices to a sense of duty''\footnote{Walden, 64}, he
responds to accusations of selfishness in his rejection of capitalism. This is
the duty that others have imposed on him, to work hard in order to earn capital
and use that capital to contribute to the greater well-being of society. In
doing so, he defies what Weber later observed as a dependence that the
capitalist order has on a ``devotion to a `calling' of
moneymaking''\footnote{Weber, 32}. That same duty is responsible for much of the
social problems he is trying to make his readers aware of, but in superficial
ways, he comes off as a freeloader, someone who manages to get by with at least
as much (if not significantly more) appreciation for life as someone who `works'
far harder. He does not dispute this; it's the most important part of his point
that he wants to get across, even at the risk of appearing selfish and
irresponsible.

The difficulty with all of this is that Thoreau already provides a rebuttal for
any criticism one could possibly give him, and builds himself a careful world in
which he cannot possibly be wrong. By already anticipating criticism, he removes
the reader's agency in reacting, such that \textit{Walden} comes off as one
long-winded ramble with no real ground for a critic to stand on. He doesn't ask
for a conversation, and he claims that he's not telling anyone how to live their
lives, but by stating his observations as if they are so blatantly obvious, he
makes people angry with the same stroke that removes their ability to defend
themselves. There is no way to argue with someone who has already set up all
possible arguments. Ultimately, the challenge he presents readers is
impossible to ignore.
