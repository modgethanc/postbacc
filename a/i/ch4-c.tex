\section{A Collective Dreaming}

\textit{14 November 2014}

The American Dream is a vague concept that combines an unavoidable status quo
with the expectation that anyone who manages to rise up from the crowd will
achieve some sort of success. In order for that Dream to stand, a harmony must
be struck between the ideal of universal self-fulfillment and the practical
needs of a collaborative society. We can witness this struggle through the
dramatization of fictional stories in which characters who try to set themselves
apart are mowed down for, ironically, trying to achieve their version of the
American Dream. On one hand, we have Edna Pontillier of Kate Chopin's
\textit{The Awakening}, whose attempt to engage with her discovery of the Dream
from a position of its infrastructure ended in failure; on the other, Linda
Loman of Arthur Miller's \textit{Death of a Salesman} presents us an option for
a woman to succeed in a context where only the achievement of men seem to
matter.

For the purposes of this analysis, I will first clarify my usage of the words
`success' and `failure'. In describing characters, I use these terms based on
their final states within the narrative work. Specifically, I refer to Edna's
ending as a failure because in taking her own life, she has made it impossible
for her to ever achieve anything further. On the other hand, I describe Linda as
having succeeded in that she survived the course of the play, and thus her
character is allowed the potential of progress beyond the narrative.  In a bit
of an oversimplification, we can also view either story as a morality tale,
through which the authors show us that characters live and die as a consequence
of how they choose to accept their places within the framework of the American
Dream.

As a concept, then, the Dream is something with both a theoretical ideal and a
practical reality.  The ideal is that all people shall achieve a status and
recognition based on the fulfillment of their human potential, as James Truslow
Adams defines in his historic analysis of the American condition\footnote{Adams,
404}. In reality, though, people still need to adhere to some sense of social
structure and organized positions, because there is no guarantee that every
person will always end up with both the potential \textit{and} desire to reach a
state that allows them to exist harmoniously with other people. That framework
forms a status quo in which all participants are unavoidably involved; Chopin
and Miller, then, explore the possibilities that lay within individual human
interactions with that framework.

With all this in mind, an obvious question appears: Why did Linda succeed, while
Edna failed? We can examine how each author set up their characters; as a
narrative device, both characters started in a situation that seemed the
opposite of their result.  Edna was placed in a life designed to succeed, with a
husband more than capable of financially supporting her\footnote{Chopin, 83
``I'll let Leonce pay the bills''} \footnote{ibid., 79, as Edna describes the
house and money as all belonging to her husband}, friends and children that
mostly adored her, and a support network of people to take care of her immediate
physical needs\footnote{ibid., 83, the maid that comes in to pick up a broken
vase} \footnote{ibid., 72, her mother-in-law and quadroon take the children off
her hands}. Linda, on the other hand, seemed doomed to fail from the start, as
her husband doesn't respect her\footnote{Miller, 64--65}, their financial
situation is in dire straits\footnote{ibid., 54--56}, and both sons seem
destined to be useless louts.

In both women's cases, though, they occupied a position within their context as
just part of the infrastructure of someone else's Dream; neither of them were
expected to engage in pursuits independent of their family unit's best
interests. At a superficial level, it seems as if the women of men were excluded
from pursuing the Dream at all, but I offer that the American Dream is one that
is pursued as a collective whole. The fulfillment of the Dream depends on the
strength of its framework, which means there is no room for someone to step
outside of their role.

This brings me to a secondary definition of `success' and `failure'; one who
succeeds has accepted their place within the American Dream, and works towards
filling just that potential and nothing more, while one who cannot accept their
position will ultimately fail in the process of trying to escape.  There's a
bitter irony in this interpretation in that the Dream itself encourages
stand-out, individualistic pursuits, but for these two women characters, their
starting points are locked into place, and their only freedom is to decide
whether or not they can accept that.

We can answer the question posed earlier by taking both of these definitions
together; Edna, as the character who failed, did not accept that she must exist
as just another leg propping up a society she could not even understand, while
Linda allowed herself to become the tireless supporter of the burden of a
failing family. The irony of both situations further highlights the tragedy of
these women's roles within the American Dream: Edna refused to accept her
success, and thus she failed; Linda unconditionally accepted her failure, so she
succeeded.

If the heroic position is occupied by a man on the path to win at life, women
exist as part of the implicit infrastructure for the American Dream. By allowing
Linda to survive through the tragedy of Willy Loman, Arthur Miller shows us that
there is a way for women to succeed in the fulfillment of the American Dream.
Chopin shows us the other option, as Edna's failure stems from the fact that she
had entered into a position that required her adherence to a framework that
depends on its own self-perpetuating existence. The American Dream is not one of
individuals, but one shared with everyone who dreams it, and it cannot support
itself without the cooperation of its component parts.
