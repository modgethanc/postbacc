\begin{description}
    \item [79-346] \textsc{American Political Humor from Mark Twain to the Daily
        Show}; \textit{Dr. Scott Sandage}---This course takes a cultural
        approach to U.S.  history since the Civil War, as seen by the nation's
        most astute and influential critics: its political humorists.  Besides
        immortals like Mark Twain and contemporary novelist Don DeLillo, we will
        (re)discover the satirical yet hilarious voices of H.L. Mencken, Will
        Rogers, Dorothy Parker, Walt ``Pogo'' Kelly, Richard Pryor, Fran
        Lebowitz, and others through essays, novels, recordings and films.
        Throughout the term, we will collaborate in defining terms and learning
        a vocabulary we can use to discuss and write analytically about
        ephemeral, topical critiques that make us laugh in order to make us
        think. How does ``humor'' differ from ``comedy'' or from ``jokes''?
        Beyond lampooning government or elections, what makes humor
        ``political''? What are the relationships between politics and art?
        What can political humor reveal that we might not ``get'' by any other
        means?  At its sharpest edges, humor addresses issues of class, gender
        and race in American life, and provokes alternative thinking about mass
        culture, consumerism, and conformity. To provide context and analytical
        resources for these themes, we will also read historical studies and
        relevant theories by Sigmund Freud, Luigi Pirandello, and Mary Douglas.
        Assignments include four analytical essays, entries in a collaborative
        online glossary, a brief oral report, and occasional short quizzes on
        assigned readings.
    \item [79-316] \textsc{Trajectories in Photography: Prehistory to 1945};
        \textit{Dr. Leo Hsu}---This course explores how photography influenced
        and was shaped by social and political changes in the 19th and early
        20th centuries. We will investigate photography in its modern and
        modernist constructions, with special attention to both continuities and
        ruptures between the pre-modern and the modern. Specific topics will
        include: the nature of pictures and precedents in picture-making, from
        cave paintings through 20th-century experiments in photography;
        photography's role in the rationalization of geographies and peoples;
        the promises of photography as a new technology alongside electricity
        and the motion picture; the position of photography in relation to fine
        art; publications, mass media and propaganda; social photography,
        documentary photography and activist photography; and vernacular
        photography and photography's popular uses. The course draws from
        various disciplinary perspectives including art history, anthropology,
        history, and science and technology studies. The course will include
        instructor lecture, student presentations, and guest lecturers. Class
        discussion will be an integral aspect of the class.
    \item [79-212] \textsc{China and Its Neighbors: Minorities, Conquerors, and Tribute
        Bearers}; \textit{Dr. Donald Sutton}---This course examines East Asian
        peoples on the periphery of the Han Chinese and their interrelations
        from the time of Genghis Khan to the present, including Mongols,
        Manchus, Koreans, Tibetans, Muslim Turks of Central Asia, and ethnic
        groups of south China. It is, in part, a history of a civilization seen
        from its margins. We question the usual narrative of Chinas
        uncomplicated absorption of its neighbors and conquerors, and pay
        attention, unconventionally, to voices of minority peoples. Besides
        ecology, war and diplomacy, we examine cultural conceptions and mutual
        influences. We also look for the emergence of a sense of identity among
        peoples in contact, including Han Chinese, especially at the onset of
        nationalism and industrialization. The course also looks at some Western
        views of the subcontinents peoples.
    \item [79-245] \textsc{Capitalism and Individualism in American Culture};
        \textit{Dr. Scott Sandage}---This small discussion course traces ideas
        about individualism and capitalism in the U.S., from colonial times to
        the present. We will focus on three main themes: 1) the relationship
        between capitalism, work, and identity; 2) changing definitions of
        success and failure; and 3) the historical origins of contemporary
        attitudes toward 1 \& 2. In short, we will study the economics and
        emotions of the American dream: how class, race, gender, occupation, and
        ambition shape our identities.  Readings include ``The Autobiography of
        Benjamin Franklin,'' studies by Alexis de Tocqueville and Max Weber,
        writings of Frederick Douglass, Ralph Waldo Emerson, Herman Melville,
        and Henry Thoreau, Kate Chopin's ``The Awakening,'' Andrew Carnegie's
        ``Gospel of Wealth,'' Arthur Miller's ``Death of a Salesman,'' and other
        works. Grading is based upon a readings journal, participation in
        discussion, three short essays and a longer final paper.
    \item [79-262] \textsc{Modern China}; \textit{Dr. Donald Sutton}---Assuming
        no prior familiarity with China or its culture, this course examines
        China's continuous changes from the 1800s on, in its cultural
        traditions, identities, daily life, social relations, and
        self-perceptions, engendered by both internal initiatives and external
        contact. We look at how changes unfolded in mass movements and in
        individual lives, in statecraft thought and in societal practices. We
        examine the roles of such historical actors as the extended family,
        modern reformers, the state, the parties and ethnic groups. Participants
        learn to use primary sources in making historical observation and to
        critique some analytical approaches to modern Chinese history. Since we
        rely heavily on assigned readings, active class participation is
        essential in this course.
    \item [79-375] \textsc{China's Environmental Crisis}; \textit{Dr. Donald
        Sutton}---In the context of China's changing ecology, this course
        explores whether and how sustainable development has been, is being, and
        might be pursued by its vast population and political leadership.
        Without neglecting culture--e.g., Confucian, Daoist, Buddhist and Altaic
        (steppe) views of ideal human/environment interaction--we trace
        historical demographic patterns and their effects on China's fauna and
        flora, and investigate past government efforts at water control,
        migration, new crop introduction, natural disasters, etc. Over half of
        the course concerns the People's Republic (1949-), paying special
        attention to birth control policies, the steppe reclamation, the Three
        Gorges dam, industrial growth, pollution scandals, tourism and
        environmental policy. We work mostly by discussion, centering on
        materials read in advance by class members.
    \item [79-317] \textsc{Art, Anthropology, and Empire}; \textit{Dr. Paul
        Eiss}--- This seminar will explore the anthropology and history of
        aesthetic objects, as they travel from places considered ``primitive''
        or ``exotic,'' to others deemed ``civilized'' or ``Western.'' First, we
        will consider twentieth-century anthropological attempts to develop ways
        of appreciating and understanding objects from other cultures, and in
        the process to reconsider the meaning of such terms as ``art'' and
        ``aesthetics.'' Then we will discuss several topics in the history of
        empire and the ``exotic'' arts, including: the conquest, colonization
        and appropriation of indigenous objects; the politics of display and the
        rise of museums and world fairs; the processes by which locally-produced
        art objects are transformed into commodities traded in international art
        markets; the effects of "exotic" art on such aesthetic movements as
        surrealism, etc.; and the appropriation of indigenous aesthetic styles
        by ``Western'' artists. Finally, we will consider attempts by formerly
        colonized populations to reclaim objects from museums, and to organize
        new museums, aesthetic styles, and forms of artistic production that
        challenge imperialism's persistent legacies.
\end{description}
