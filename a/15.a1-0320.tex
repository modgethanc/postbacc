Vincent Zeng
Professor Donald Sutton
76-262 Modern China
20 March, 2015

The Adaptable Prosper

China in the early 20th century was a chaotic period, especially for the
intellectual class of the late Qing dynasty.  The Confucian system that
preserved a national sense of stability had crumbled, leaving a wake of
abandoned ideologies and ambiguous institutional structure. As traditional
expectations for success and prosperity proved unsustainable, some members
of the former elite turned to business ventures as a means to survive.
However, the world of capitalism is a much different one than the feudal
system of the Confucians; those who could adapt to the rapidly changing
order were more likely to make it.

The collapse of the Qing dynasty was, in part, an effect of foreign
influence which destabilized the interior. Most of this influence came
in the form of trade, which brought with it strong pressures to
modernize and accept a more global economic order. We can witness this
by observing the trajectory of the Ye family branch in Tianjin, as
Esherick describes changes in that city as a treaty port (120).  Between
the establishment of foreign concessions, rapid development of modern
public infrastructure, and heavy commercialization, Tianjin became an
area well-suited for incubating the new Chinese businessmen. Ye Chongzhi
was thus placed in an ideal position for this transition, due to his
previous experiences in industrial management and his convenient
location for accessing economic and trade centers. Perhaps he sensed
that the instability of the government would not provide him a
sustainable career, and thus made “the provident decision to abandon
politics for business” (Esherick, 124). In the early 1900s, with
frequent connections to outside affairs, Ye Chongzhi got a head start on
the sort of adaptations the rest of China would have needed to make.

Further inland in Shanxi, the effects of modernization moved a
little slower, but were still noticeable. In 1913, Liu Dapeng gave
up his teaching position and entered the more profitable coal mining
business for the sake of being able to support his family (Harrison,
113—114). Despite the fact that the Confucian-based system of the
Qing era had constantly let him down, though, Liu approached doing
business with the same moral code that carried him through his past
failures. He frequently took on mining ventures out of a sense of
altruism and good will, as evidenced in his takeover of the Shimen
mine in order to help a struggling family turn some profits
(118—119). Members of the community knew they could rely on him for
being honest and trustworthy, attributes that were critical for
developing and maintaining a positive reputation in an otherwise
chaotic political and economic environment (122—123).

Despite his earnest approach to business, there were many
changes outside Liu's control that made it difficult for him to
prosper. Outrageous taxation became a heavy burden in Shanxi,
especially as local politicians started vying for more power and
leaned on local residents to support their campaigns (131).
Technological advances in mining equipment made a mess of the
mountainside and cause the economy to favor large commercial
ventures, rather than the smaller pits that Liu was involved
with (134). In the face of these changes, Liu held on to his
strong moral code, as this integrity was the only thing he knew;
he was thus unable to adapt to the heavily profit-based biases
of the mining business, and slipped further into poverty over
the next couple of decades (135).

By the 1930s, we see an entirely different approach to doing
business in China. The Rong family, which established
numerous textile and flour mills just before the outbreak of
war, demonstrated a wider range of adaptability to changing
political structures and economic pressures. This
adaptability, above any other strong ties to particular
ideals, was what allowed them to persist through an
otherwise turbulent period. Early on in their ventures, they
noticed that battles with the Japanese near their factories
in Shanghai threatened their business; they keenly observed
the effects of foreign presence, and counted on the fact
that their properties near the concessions would be
protected (Coble, 115). Once the fighting ended, patriarch
Rong Zongjiang clearly recognized that  the Chinese
government would not protect them, as evidenced by his
formation of a merchant alliance that eschewed ties to any
political entity (116—117). His bold statement that “China
is virtually a nation without a government” (117) showed how
little faith he had in politics to protect the well-being of
the people.

This divorcing of the family business from national
loyalties goes further than just forming an organization
exclusively to protect commercial interests. After
Shanghai fell to Japanese forces in 1937, the Rongs
registered their mills as foreign American and British
ventures, relying on an alliance with entities that the
Japanese did not yet wish to disturb. After Pearl
Harbor, this alliance was no longer valuable, and in
fact made those mills more of a target; the response,
obviously, was to revert to Chinese registration to
attempt a resistance against Japanese seizures (125).
Although that action failed to prevent a Japanese
takeover, it showed the Rongs' willingness to have a
flexible national affiliation for the sake of protecting
their bottom line. Ultimately, they regained control of
their mills by going along with Japanese efforts for
economic cooperation, essentially playing to their
desires for Pan-Asian unification (127).

Beyond a flexibility in political dealings, the Rong
family also played to fluctuations in economic
demand to maximize potential for profits. During the
decline in national production due to the war, they
reduced  loss of profits by running their equipment
in small rural productions, which gave them the
benefit of being able to evade government
regulations that larger operations were subjected
to, while also granting them access to raw materials
when foreign imports slowed (130). Their mills in
Wuhan prospered during the war because they were
able to secure good local sources for cotton and
wheat, combining the output from both their textile
and flour mills to provide much-valued bags of flour
at a time when finished products were highly
desirable (133). Through such methods of commodity
hoarding and playing to the fluctuating desires of
the market, the Rongs managed to survive through the
hard times of war, while less profit-driven ventures
were unable to keep up.

In studying the Rongs' administration of their
business, we find little evidence of the
Confucian value system of loyalty and avoidance
of wealth-gathering that dominated the previous
generation of Liu Dapengs and Ye Chongzhis.
While much of the Rongs' success through that
period was a result of the larger scale of their
enterprises and a good amount of luck, their
willingness to adapt their business practices to
do whatever it took to survive played a
non-trivial role in allowing them to prosper.
During a time when old systems have failed and
new systems have yet to take root, adaptability
triumphed over tradition.

References
Coble, Parks. 2003. “The Rong Family Industrial Enterprises and the War”. Chinese Capitalists in Japan's New Order.
Esherick, Joseph. 2011. “Doing Business in Tianjin”. Ancestral Leaves.
Harrison, Harriet. 2005. “The Merchant”. The Man Awakened From Dreams.

