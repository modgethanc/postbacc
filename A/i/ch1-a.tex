\section{The Greyness of Mark Twain}

\textit{18 September 2012}

It is a tempting and simple instinct of humans to divide the world into a series
of dichotomies; truth and lies, civilized and savage, sin and morality, black
and white. However, it is the spectrum between, the infinite divisibility of the
grey area, that provides truly worthwhile things for study. The extremes only
exist as notions, ideals that can never be realized, for one requires the
context of one side in order to note the existence of the other. Acknowledgment
of both reveals a vast and muddled space between, in which travel becomes
possible. Throughout his life, Mark Twain embarked on his own journey through
this inexplicable grey area, and his discoveries formed the basis of his humor.

In \textit{The Importance of Mark Twain}, Alan Gribben defines Twain as the
quintessential American humorist, noting that he exists as a "crucial
continuity" and a "common denominator of what we want to perceive to be the
American character" (Gribben 48). Indeed, Twain's status in history stands as an
almost mythological figure, a strange life bookended by two appearances of an
astrological phenomenon. Charles Neider describes him as a living incongruity,
both a clown and a tragedian, both an amateur and a competent professional, both
an optimist and a pessimist (Neider xv). Twain's inexplicable connection to both
the blacks and the whites of the world, an intimate understanding of the
existence of dichotomies, allowed him to display the gradient of greys as he
dilutes whiteness with black and vice versa.

Twain begins his foray into mass publication with an open editorial to
\textit{The Buffalo Express}, in which he declares from the start that he shall
be existing in the grey field. In \textit{Salutatory}, he provides the paper's
readership with a list of absolutes, such as confining himself to truth,
rebuking all forms of crime, refraining from vulgar speech, etc. except when he
won't (Neider 1). This is his initial declaration to the literate public at
large, his claim that he will not be adhering to canonically accepted notions of
custom and  law. In this early period of his life, he has already accepted the
grey area that will become the color of his future writings; he starts with the
pure whiteness that characterized publications before him and rejects it by
adding blackness, pushing his position into issues concerning the space between.

Twain found good company in another figure that occupied a similarly incongruous
state. He displayed a certain fascination with the late King Kamehameha V when
he wrote \textit{The Sandwich Islands}. There, he outlined the existing
dichotomies of Catholic versus heathen, civilized versus savage, the incoming
whites versus the existing natives. With those categories of white and black
established, he then immediately showed his appreciation for Kamehameha's
ability to move between them, how he could "converse like a born Christian
gentleman" and then "retire to a cluster of dismal little straw-thatched native
huts by the sea-shore, and there for a fortnight he would turn himself into a
heathen whom you could not tell from his savage grandfather" (Neider 25).
Twain's tone is a bit whimsical, almost envious, of Kamehameha's flexible and
confusing status; he wrote with obvious distaste and sadness that such a place
would be annexed by America, through a process that was seemingly bringing the
"lamp of light" into the darkness (Neider 28). In reality, Twain found that
something to be ridiculed, as he sarcastically exclaimed, "We can give them
lectures! I will go myself." (Neider 28).

William Keough portrays the particularly American humor that Twain frontiered as
a more violent, caustic sort than that of other cultures. Violence, however,
exists on a more basic human level than humor, and humor as a response to
brutality is a way of removing the blackness of it, bringing a dark situation
into the grey field so that it can be more easily viewed in a context with
contrast. Twain addressed a "Person Sitting in Darkness" from the position of
one who occupies the grey area; he did not write from the lofty tone of one who
would banish the darkness entirely by shining the full power of the sun into the
shadows, nor does he wish to snuff out the light entirely by covering it with
overwhelming shadow. Rather, he straddled the middle, addressing both the voice
of light and the listener in the dark.

\textit{"To the Person Sitting in Darkness} is a dark, bitter essay, one that
can hardly be considered humorous so much as it merely contains some elements of
humor. Sarcastic comments directed towards Reverend Ament, for example, "the
right man in the right place" (Neider 284) serve to muddy the normally pure
white image of a man of the cloth; at the same time, sympathetic speech to the
listeners in the dark, "too scarce and too shy" (Neider 286), bring them out of
the darkness in which they have otherwise been placed. He levels the playing
field for them by causing both sides to exist in the same field of grey, one
that allows a closer relate-ability once they are all just humans in the world
together rather than strict representations of black and white.

He goes further to constantly point out dichotomies in order to highlight the
hypocrisy that occurs when one attempts to adhere to them; his Person Sitting in
Darkness begins to learn of the incongruities that Twain denigrates when two
kinds of Civilization are pointed out, an externally displayed one, "strictly
for Export", and an "Actual Thing that the Customer Sitting in Darkness buys"
(Neider 287). The Person is given a voice of his own once he starts to describe
the incongruities of so-called civilized societies "with its banner of the
Prince of Peace in one hand and its loot basket and butcher knife in the other."
(Neider 289).  The Person continues to ponder such things for Twain, noting that
it is "curious and unaccountable. There must be two Americans; one that sets the
captive free and one that takes a once-captive's new freedom away from him, and
picks a quarrel with him with nothing to found it on" (Neider 291). Twain's
ultimate point is that so long as such dichotomies exist, there will be
incongruities, as it is impossible for man to remain firmly in either the black
or the white.

The essay's bitter dissection of America's habit of violently conquering native,
heathen people comes to a conclusion that is explained to the person; in
particular, he states that "we have debauched America's honor and blackened her
face before the world" (Neider 295). America has been made a muddied grey
through its violent actions, bringing the country into a hypocritical and ironic
state that is ripe for Twain's humor to feed upon. The joke is that there isn't
a joke, but couched in Twain's sarcastic tone, the mixing of black and white
becomes an object of ridicule, a situation that necessitates a humorous
interpretation in spite of itself.

Once he travels too far into the black, Twain begins to lose his edge of humor.
In \textit{The Damned Human Race}, he writes almost entirely from the black,
reaching a point in his travel across the grey space from which he does not see
a recovery for mankind. He declares that "Man cannot claim to approach even the
meanest of the Higher Animals", and "constitutionally afflicted with a Defect
which must make such approach forever impossible, for it is manifest that this
defect is permanent in him, indestructible, ineradicable" (DeVoto 228). By both
placing man in an unrecoverable darkness and situating himself in a position
that no longer acknowledges the grey area, he wrote a piece that does not
provide or a way out, nor any sympathy for humanity. In this, the absence of
humor is apparent.

To examine Twain's spectrum of writing is to see a full range of the grey
spectrum, from his early pieces that add subtle amounts of greyness in order to
give some contrast to the world of black and white absolutes, to essays that
spell out the grey area entirely, and the final blackness of his later writings
that stand as the opposite to his initial editorials. The parts of humor that
are visible are ones where the grey area is clearly visible, when both the black
and the white boundaries can be seen, and an occupation of the grey space is the
only position from which that view is possible.
