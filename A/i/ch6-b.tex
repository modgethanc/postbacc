\section{For the Sake of a View}

\textit{24 April 2015}

In May 2014, I visited the Tianzhushan Geopark in Anhui. The area is currently
under tentative UNESCO World Heritage consideration, with the most recent
submission in early 2015\footnote{UNESCO, WHTL-5992}, and has had status as a
national scenic spot since 1982 \footnote{Huang Wen, Tianzhushan summary}.
There are no roads past the foothills just outside the city of Anqing; a set of
cable cars provide transportation for visitors and hospitality workers from the
entrance of the park to areas closer to the peaks. Beyond the cable cars, stone
steps constructed over many decades of park management lead visitors to the best
vistas, interpretive areas, and restaurants. Modern hotels, with flat-screen
televisions in every room and large aquariums in the entry way, had their
components carried to the mountain piece by piece by cable car, according to the
receptionist where we stayed.

My mother had visited the park in 2012, and noticed that cable cars she
previously ridden had vanished. When she asked a park worker about it, he
informed us that the cable cars were ruining the scenic vista by interrupting
the natural appearance of the distant granite peaks (``\textit{po huai
fengjing}''). If we looked closely, we could still see the gap in vegetation
that had not yet grown enough to cover the hole bored into the side of the
mountain for the old cable car boarding areas. My uncle agreed with the park
worker that wires hanging in the sky disturbed his view; he was having a hard
time taking photographs without some evidence of Chinese overpopulation crowding
into his frame, because at that level, the power lines that kept the lights on
in our hotel were still visible. Of course, we'd have to climb further on carved
stone footpaths if we wanted a more pristine view.

This perceived value of a scenic mountain vista causes a paradox in the
management of parks; the subjective aesthetic qualities of a particular land
feature gives it some sense of appeal for the visitor, but as pressures increase
from a growing tourist population, the infrastructure must grow to accommodate
it. Tourists need places to stay, food to eat, restrooms to wash; hospitality
workers and park staff need transportation to and from their work location. As
this infrastructure grows, so does the risk of damaging the aesthetic value of
the park. The problem of balancing tourism infrastructure with tourism value
doesn't end at the opposing desires of an individual visitor to have both a
comfortable experience and an unobstructed view; rather, it extends into the
foundation of a park itself, affecting the motivations of park management, and
forming a cyclical reaction.

To better understand this value system, we can compare self-evaluations of parks
located in China with the same in the United States; in both cases, the
evaluations were compiled by internal committees involved with parks management
as a response to reporting standards for maintaining UNESCO World Heritage Site
status. Specifically, I present two sites for study: Wulingyuan Scenic Area, in
Hunan, and Yosemite National Park in California. Both sites annually record high
numbers of visitors whose primary activities are leisure and sightseeing, as the
parks encompass relatively unique geological features which are of great
interest to tourists; managing the impact of those visitors is one of the
greatest challenges of either park. Additionally, the UNESCO designation
requires each park to answer to a set of standards maintained by an outside
party; differing responses to this external force provides a sharp sense of
contrast between each value system.

\subsection*{Wulingyuan Scenic Area}

In 1992, Wulingyuan achieved status as a UNESCO World Heritage site, under
criterion (iii) for natural sites. The original inscription used language such
as ``undeniable natural beauty'' and ``outstanding sightseeing
value''\footnote{UNESCO, Wulingyuan Periodic Report, 3}, phrases which are
subjective judgments that speak to very specific cultural values. One has to
understand and agree that ``jagged stone peaks, luxuriant vegetation cover and
clear lakes and streams''\footnote{ibid.} are features that contribute to the
inherent value of the site in order to accept its classification as a world
heritage that must be protected. While Wulingyuan certainly had some value
within its original context prior to UNESCO involvement, it's clear that the
designation has affected further management and policy decisions regarding the
site. In a direct response to critical UNESCO evaluation comments in 1998, the
committee managing the park thanks UNESCO for its ``timely, objective, fair, and
practical evaluation and criticism'' that showed the committee ``the importance
of restoring the place to its original heritage value''\footnote{ibid., 7}.

The periodic report continues to detail past actions, current status, and future
policies moving forward with the management of Wulingyuan, almost as a defensive
statement against the 1998 inspection. Most UNESCO periodic reports do not seem
to require such detailed statements, as the main reports are just to respond to
the UNESCO-provided questionnaire on park status. Nevertheless, the committee
acknowledges, in accordance with the original inscription, that the sandstone
mountains are of utmost value, and they must ``protect [the] genuineness and
integrity of the peaks''\footnote{ibid., 8}. It goes without saying that
quarrying and damaging the vegetation are strictly forbidden; however, they
needed to drill through the rock to perform ``necessary construction'' in order
to install cable cars, elevators, and other supporting infrastructure ``to offer
convenience to tourists''\footnote{ibid.}. In other words, damage to the
mountains for certain reasons (the mining industry, agricultural needs, etc.)
are not permitted, but damage that contributes to appreciation of the mountain's
value as an ``undeniable natural beauty'' is acceptable.

As they are charged with protecting the `heritage' of Wulingyuan, the report
committee detailed their process from 1992 to 2002 to plan and execute such
measures. This included setting up an initial `Heritage Office' in 2000, with
only three full-time employees that answered to the Wulingyuan Scenic Resort
Administrative Bureau, contracting the City Planning Center of Beijing
University for 6 million yuan to prepare a management plan in 2001, and
increasing to four separate departments dedicated to heritage protection by
2002. This addition of resources brought the park staff to 500 employees, whose
mission was to ``crack down on any acts that destroy heritage
resources''\footnote{ibid., 12}. These measures were a direct response to a
UNESCO site inspection in 1998, which described the area as ``overrun by tourist
facilities, having a considerable impact on the aesthetic qualities of the
site''\footnote{UNESCO, 1998 State of Conservation reports, 14}. The increase in
annual visitors grew from 350,000 in 1992, the year of inscription, to 1.5
million in 1998, when the inspection was performed\footnote{UNESCO, Wulingyuan
Periodic Report 36}; it seems inevitable that the park would experience a growth
in `tourist facilities', as more infrastructure was necessary to support this
rapid growth.

In addressing the problems outlined in the 1998 inspection, the committee
acknowledged that Wulingyuan is a ``precious resource which has evolved after
[a] million years of development'' that had only maintained its value because
``people had no access to the area''\footnote{ibid., 13}. As a result of the
uncontrolled development, though, they worried specifically that ``the aesthetic
value of the heritage is being spoiled''\footnote{ibid.}; they vowed to take
immediate action to ``restore the heritage to the original appearance'' by
removing all buildings and residents from the scenic area\footnote{ibid.}. From
1999 to 2002, they demolished around 200 buildings (a mixture of reception
facilities and residential dwellings), relocated 377 households, and cleared
around 175,000 square meters of land, which cost about 66 million yuan. However,
these buildings presumably sprang up as locals sought to take advantage of the
growing tourism industry; the cable cars and restaurants that catered to
visitors were culprits for destroying the scenery (\textit{po huai fengjing}),
but, ironically, were only there because of the park's popularity in the first
place.

The park's response to simply demolish buildings and kick out profiteering tour
guides seems like only a knee-jerk response to the 1998 report of compromised
scenery; this emphasis on the aesthetic value of the area dominates all other
notions of preservation. It's clear that they are having a hard time keeping
local tourism entrepreneurs out of the park; the report claims three separate
attempts to remove residents from the area, without apparent success. Their
compromise, then, is to permit a few households to remain as ``a picture of
bucolic life'', living under carefully controlled regulations that forbid them
from starting commercial tourist ventures, or use pesticides and fertilizer in
their fields\footnote{ibid., 20--21}. If having unsightly shacks is truly so
unavoidable, then their response is to redefine the aesthetic to include local
residents, integrating the image of rural farmlands into the park's ``original
beauty''\footnote{ibid., 21}.

Despite the rapid increase in tourists causing a negative impact on the quality
of the scenery, the Wulingyuan District still promotes the site through
guidebooks and advertisements\footnote{ibid., 16}, and annual visitors have
increased from about 1 million in 1996 to  about 4.5 million in
2001\footnote{ibid., 36}.  Although it's difficult to attribute this four-fold
growth to anything specific, it's clear that such a significant growth is
straining the park's resources. Curiously, the report claims to have ``managed
to maintain a stable number of tourists'' in the same section as acknowledging
visitor growth and the need to control this traffic\footnote{ibid., 37}.
Overall, the report presents a confident tone in how well the park has been
managed, which belies growth factors and illegal residents as issues that are
clearly not under control.

\subsection*{Yosemite National Park}

In contrast to the Wulingyuan report, the presentation of the Yosemite National
Park periodic report of 2013 is far more terse, giving exactly the information
required by UNESCO without providing excessive justifications, defenses, or
promises. Nominated in 1984 under natural criteria (i) for ``distinctive
reflections of geological history'' and (iii) for ``exceptional natural
beauty''\footnote{The periodic report of 2013 uses revised criteria (viii) and
(vii), respectively, but I use the original designations here to avoid confusion
in comparison to the Wulingyuan site.}, Yosemite stands under a similar pressure
to preserve a particular aesthetic value. In his history of Yosemite, Richard
Grusin simplifies the wider goal of America's national parks as ``the desire to
withdraw or preserve particularly spectacular natural areas from the threat of
social, political, and economic development''\footnote{Grusin, 1}; put in a
slightly different perspective, separating humanity from Yosemite was a part of
the park management's goal from the beginning, so it was not necessary to
emphasize that goal in the UNESCO report. The preservation of this aesthetic is
such a central part of the American identity that it's completely taken for
granted, and thus doesn't need to be mentioned.

The 2013 periodic report presents a chart of potential negative factors on the
park, using adjectival descriptions to outline the impact and progress on
categories such as development, infrastructure, waste, and
climate\footnote{UNESCO 2013 Yosemite Periodic Report, 3}. Unlike the Wulingyuan
report, this information does not have responsive actions described inline; it
is simply a list of factors with potential effects on the park's status as a
World Heritage site. In following the periodic reporting format, the
``protective measures'' appear on a different page, and cite the precise
department responsible for management, as well as specific policies that deal
with park conservation\footnote{ibid., 5}. It's specified that Yosemite is ``the
first scenic natural area to have been set aside for public benefit and
enjoyment'', as publicly-owned land under the legal jurisdiction of the federal
government\footnote{ibid.}.

While the report does not provide numbers for visitor traffic, it does mention a
``slight increase'' in annual visitation over the past five
years\footnote{ibid., 9}; without being able to quantify this increase, it's
difficult to make a comparison to the Wulingyuan visitation (though we can
reasonably assume it's far less than the four-fold growth Wulingyuan experienced
in a similar time slice). However, heavy traffic does not seem to present a
growing problem in the preservation of the park. Referring back to the chart of
negative factors, ``impacts of tourism/visitor/recreation'' is described as
``significant'' but with a ``static'' trend, and effects of buildings,
transportation infrastructure, and pollution are either ``decreasing'' or
``static''\footnote{ibid., 3}. As a reflection of park management, this seems to
describe an inevitable amount of tourist traffic, which is mostly under control
by park management, and under continuing efforts to reduce impact.

The tone of the Yosemite report reads as if the committee is settled, confident
in its policies, and stable in the management of the park. This is not
necessarily a consequence of having more experience with writing UNESCO reports;
the formation of the park from the start already dictated the sort of
relationship that particular place would have with the people who visit it.
Other than direct references to the UNESCO designation, the report rarely
mentions beauty or aesthetics as a motivation for any of the policies at all.
It's as if the aesthetic value is taken for granted as an inherent part of the
park, and thus not necessary to mention in a report for external validation. As
far as this report is concerned, Yosemite is a stable, well-maintained site that
is set up to continue that status in perpetuity.

The UNESCO report does not adequately reflect some of the history of Yosemite
National Park; the head wall of El Capitan stands as an iconic feature of the
American wilderness, one that has challenged artists, adventurers, and advocates
since it was shown to the Western world in the 19th century. This timing was key
in its preservation; Jen Huntley credits the rise of public knowledge of these
places as ``sites of divine redemption and antidotes to the problems and
stresses of modernization''\footnote{Huntley, 3}, and that the features that
contributed to those sites holding their power over time were the very same
things that people sought to avoid. In other words, the desire for pristine,
untouched wilderness in the American West comes out of a desire to escape some
crushing facet of human society, while the value of those views are inextricably
tied to consequences of human society. Yosemite was valued because it looked
untouched, and continues to have value because keeping its untouched appearance
is part of its infrastructure.

\subsection*{Implications}

The Wulingyuan and Yosemite periodic reports are both documents prepared
internally, to be presented to an external agency. Why is it that the Wulingyuan
report comes off as so defensive, with frequent reassurances and references to
the terminology of aesthetics, while the Yosemite report is cut and dry, showing
little insecurity about the management's ability to keep the park stable?
Perhaps it is exactly that; Yosemite National Park can remain relatively stable
because the aesthetic context in which it exists is fairly well-established, and
the Wulingyuan District is still adjusting its value system to account for
UNESCO-defined aspects of ``undeniable natural beauty''.  Yosemite already has
value as, by Huntley's description, ``a `sacred' space, separated from if not
antithetical to the workings of market capitalism''\footnote{Huntley, 80}.
Though paradoxical by nature, this value still demands that humanity and
Yosemite remain conceptually distinct, and thus maintaining a sense of the
sublime, untouchable wilderness.

It would be too harsh to say that the Wulingyuan District's sense of natural
beauty is incompatible with UNESCO's expectations; rather, this speaks to a
discrepancy in value systems that just make it difficult for the two parties to
communicate their intentions. At the most basic level, it's a failure rooted
first in the language itself, and secondly in the implications behind that
language. In studying the Chinese relationship with the Western ideals of
nature, Robert Weller points out that the modern, commonly-used phrase
\textit{daziran} is a rough translation of the English word `nature', brought
into the Chinese consciousness in the twentieth century\footnote{Weller,
20--21}. However, there is no concise word that easily refers to `nature' with
all the connotations that a Westerner takes for granted; one can describe it in
a roundabout way, like `outside' (\textit{waimian}), `mountains and water'
(\textit{shanshui}), `wild lands' (\textit{yedi}), and so on, but those terms do
not call to mind the sublime landscape of Yosemite when one thinks of simply
`nature'.

The notion of a scenic spot (\textit{fengjing}), on the other hand, has recorded
use as far as 200BC\footnote{Nyiri, 7}. The visual impact of \textit{fengjing}
was carried by artists and poets through 16th century travelogues, which were
followed by wander literati, and eventually compiled in 20th century
gazetteers\footnote{ibid., 7--9}. The \textit{fengjing} refers almost
exclusively to the appearance and shape of the view; how separated it is from
evidence of human habitation does not play a strong role in its value (except,
of course, when human structures visibly alter the accepted canon of that view).
In comparing the two UNESCO reports, then, we can perhaps attribute the
reactions of the Wulingyuan committee to a struggle in aligning their idea of
\textit{fengjing} with the Western expectations of ``undeniable natural
beauty''.

Regardless of why people value Wulingyuan, it's still an undeniable fact that
the rapid growth of visitors poses a substantial problem to the preservation of
the area. This issue is not unique to Wulingyuan. Yellow Mountain in Yunnan grew
from 50,000 to 3 million between 1989 and 1999\footnote{Sofield, 155}; in that
same time frame, the total number of domestic tourists in China grew from 240
million to 719 million\footnote{Ghimire, 90}. This growth can be attributed
mostly to a rising middle class, forming the social and economic culture
necessary to promote tourism and leisure. The question for park management,
then, is how to preserve the \textit{fengjing}, without placing great
restrictions on access.

One certain benefit from increasing visitor numbers is a growth in revenue, and
this income feeds back into the park for preservation. But as a scenic area
holds potential for tourist income, it's often the case that the same location
has rich resources for industrial development. Ralf Buckley makes a few rough
estimates in comparing the income from outdoor tourism with that of using the
same area for industry; specifically, he breaks down the economics of river
tourism and pits it against hydroelectric power. His estimates include per-trip
costs for existing rafting tours on the Yangtze, and basing a theoretical number
of participants by comparing current numbers to the more mature river industry
in the Grand Canyon of the Colorado River\footnote{Buckley, 60}. With a similar
amount of conjecture, he calculates the value of future  dam projects proposed
for the same section of the Yangtze, while ignoring operating costs and assuming
infinite demand\footnote{ibid., 61}. The conclusion of his back of the envelope
calculations is that a theoretical tourist-exclusive use of the river could
bring 4.2 billion RMB per year, while hydropower might only bring in a 2.5
billion RMB return after investing in the infrastructure\footnote{ibid.,
61--62}.

Obviously, this rafting vs. hydropower thought experiment is not likely to cause
any groundbreaking changes in how the Chinese approach environmental tourism.
Buckley finds another use for tourism in China, illustrated by Last Descents, a
river guiding company that combines outdoor recreation with activism. The
company was founded as a collaboration between an American kayaker and a Chinese
businessman, named as a play on the river adventurer's motivation to achieve a
first descent on a particular segment\footnote{ibid., 58}. Last Descents
targeted rivers that were threatened by dam projects, hoping to inspire
influential clients to protect those regions by giving them a direct experience
of the natural beauty and power of the rivers.

There's no substantial evidence as to whether or not Last Descents has had any
effects on altering the usage of the rivers, or in communicating a particular
idea regarding the aesthetic value of untouched nature. However, it further
illustrates the point that there are distinctions between Chinese and Western
ideals of the perceived value of scenery; the conflict between harnessing the
rivers for hydropower and capitalizing on the rivers for tourism reflect a
question of which usage is ultimately more valuable for society.

\subsection*{The View Persists}

At Tianzhushan, it was clear to me that the park had gone through many, many
revisions. We walked on paths made from large stone blocks that were carried to
the mountain in the past decade, but they were laid next to old paths carved
directly into the rock in the 1980s. New stone railings, carved to mimic the
shape of the rock itself, sprouted over rusty scars where old iron fences were
torn down. Everyone I spoke to gushed about how much improvement the park made
every year; my 80-year-old grandmother, for example, ascended to the terminus of
the highest path at the base of Tianzhu Peak, with the help of the old cable
cars. The changes were all geared towards optimizing the view, and building more
comfortable ways to bring people to that view. Downwind of the defunct cable car
station, a constant cloud of dust rose out of the construction site for an
upcoming visitor's center. ``Why are they building more hotels?'' complained my
uncle, of course, as we descended the path towards the noisy machinery. ``The
one we stayed in was practically empty.''

Overcrowding never seemed to be a problem at Tianzhushan in the way it's
described at Wulingyuan; our hotel was indeed deserted, and there were more
service workers than visitors waiting in line at the cable cars. Above the
immediate area of the active station, very few tourists continued walking the
mile or so of steadily-climbing steps to the base of Tianzhu Peak. With fewer
convenient accommodations than Wulingyuan, Tianzhushan is unlikely to experience
the problems of an unsustainable visitor increase; why, then, has it been under
so much construction and aesthetic revision?

Perhaps the answer lies in Tianzhu Peak's status as \textit{fengjing}. A record
of ascents goes back as far as 106BC by Emperor Wu of the Han Dynasty, and
cultural relics depicting the unique mountain profile date back 6000
years\footnote{Huang}. Inscriptions carved into the side of the mountain by
historical figures can be seen from neighboring peaks. There's a draw to that
view, not because it's a pristine, untouched fragment of nature that one can
visit to escape the crushing pressures of modern life, but because it's a view
that has been altered and redefined through a long line of people, and persisted
through countless changes of authority. In other words, it's valued because it's
been touched by society, and it's accepted that the history of all the feet that
passed through it have added to that value. The endless revision and
construction, while an attempt to strike a balance between accessibility and
compromising the view, are, paradoxically, part of the \textit{fengjing} itself.
