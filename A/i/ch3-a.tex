\section{The Disputes of Grey Areas}

\textit{10 February 2014}

An examination of boundary areas often provides insight into issues that occur
on one side of the divide or the other. The fuzzy transitional zone between the
grasslands used by mobile Mongolian herders and China's stationary agricultural
settlements is one instance of such a space.  The physical landscape of that
area is characterized by a combination of constantly shifting ecological borders
as the sand dunes of the desert-steppe environment encroach onto arable
farmlands.  This changing landscape created disparate ways of life that
engendered different cultural ideologies towards the usage of space. One group
was the pastoral nomads who valued the flexibility of following the lands that
provided for their sustenance; the other, sedentary farmers whose carefully
organized cultivation of the land required a clear delineation of the world
around them. Ultimately, the proximity of these two disjoint cultures
demonstrated some of the problems that came from such an adjacency.

The Han had a sense of space that defined emptiness as a negative
concept\footnote{Williams, 671--672; connotations of the terms `huang', `ye',
etc.}, an attitude that was reflected in the very language\index{language} they used to refer to
bounded versus unbounded areas. As a civilization that frequently needed to
defend itself against invading forces, they developed a culture of maintaining
walls and explicit boundaries in order to better identify the separation between
themselves and some sort of other. Since it is difficult to fight against the
nebulous and unbounded void, empty space became a fearful object, something that
the Chinese needed to harness and tame in order to maintain control over their
borders, both physical and psychological. As a result, the very existence of
delineated space became a marker of Chinese sovereignty, reflected in highly
structured urban planning, border defense, and agricultural practice.

As a result of  these deep-seated needs for the explicit delineations of space,
unbounded territories seemed intolerably disorganized to the Han; disorder meant
a lack of control, a lack of control meant unproductivity, and unproductivity
was wasteful.  Thus, they saw the unmodified grasslands to the north as an area
of waste, but also containing potential, capable of conversion to productive
work through a process of land cultivation and the civilizing of its
inhabitants.  Regardless of the actual usage of that land, the Han needed to
organize it into something that they recognized as valuable.

Furthermore, the fact that such wasted land was occupied by other peoples proved
a challenge to the Han. It was difficult for them to view the nomadic Mongols as
anything other than aimless wanderers of the grasslands, taking up space that
could otherwise be put to better use if they instantiated a systematic
cultivation of the area. Since the herders were benefiting from land that they
had not worked to cultivate themselves, the Han saw their existence as an
affront to their values of labor\footnote{ibid., 672--673; discussion the Han
ecological identity}. In other words, they thought of the Mongols as primitive,
lazy squatters, and the way to fix this problem was to impose order onto the
chaos by organizing the wandering nomads into a structure that could produce
measurable amounts of work. Through this structure, the Chinese sought to
demonstrate their status as a more modern, developed civilization, bringing all
people under its rule into a high standard of organized and productive living.

From those attitudes, the Chinese government made a series of policy decisions
to impose a Han-centric sense of structure and organization onto the northern
grasslands region. The primary effect of these policies was to decollectivize
the Mongol way of life; where the herders once shared the open grasslands as one
continuous area, the land was instead divided into discrete parcels. An
enclosure policy was instituted to prevent livestock from grazing freely on
areas that the Chinese agricultural scientists identified as vulnerable to
erosion, as well as provide a system to track the work value of each parcel.
The smaller plots of land could, in theory, be more easily monitored for levels
of productivity and stop potential damage resulting from overgrazing of the
grassland. Livestock and plots of land were then assigned to individual
households depending on the number of animals a particular plot could support,
and families were given incentives to adhere to these boundaries\footnote{ibid.,
680; enclosure policy details}.

As part of the enclosure policy, households that were more productive were given
subsidies by the central government. Though the subsidies were intended to
encourage adherence to the new order of cultivation, they had the side effect of
creating a wealth gap, as families that produced more could buy more fencing
materials, secure their own plots against their neighbors' livestock, and let
their own animals graze on unsecured land. So, the policy in practice allowed
households to enclose their land to stock up good grazing areas, saving that
grass for when the rest of the pastures grew thin. In addition to causing more
land erosion by allowing the open grazing that the enclosure policy was supposed
to restrict, feuds often resulted from disputes over unclear property lines and
animals that strayed too far into the wrong plot\footnote{ibid., 684--685; some
of the practical responses to the appearance of fences}.  Eventually, earning
work points and subsidies became the main motivating factor for the Mongol
herders who wanted to survive; in order to protect their own interests, each
household had to find ways to be more productive than their neighbors, rather
than tend to their flocks in the cooperative traditions.

Clearly, such sweeping changes could not have happened without resistance, as
was demonstrated by the students in \textit{Wolf Totem}. They grew to sympathize
with the Mongol way of life and lamented its destruction by Chinese policies.
When the main characters returned to the grasslands after being away for thirty
years, they were struck by the changes of the landscape when they saw fenced-off
and untouched grass pastures, dried rivers, and ragged-looking
animals\footnote{Jiang, 507--511 of \textit{Epilogue}}.  The effect of the
enclosure policy was obvious to them; parcelization of the land had disrupted
the normal grazing patterns that had allowed the grasslands to remain
sustainable. Chen commented bitterly on the transformation and noted the
short-sightedness of sacrificing the land for a quick economic boost, since he
understood that ``the grassland could not return even if the subsidies continued
for the next century''\footnote{ibid.,, 510}. In their time  of living with the
Mongols, they knew that the survival of the grass was the key to the survival of
the people, so when the land suffered, so did its inhabitants.

Beyond the ecological disruption that Jiang's characters condemned, they
witnessed a cultural shift that came from the imposition of the new Chinese
order. As youths, they drew a parallel from the differences between Mongol and
Chinese horses to the differences between Mongol and Chinese society. Even
though they were Chinese from the city, they recognized and valued the spirit of
the Mongol horses, which they considered an untamed strength that could defeat
even the greatest acts of labor. They called the Chinese workhorses ``stupid'',
and regarded them as slaves that could only work and provide service to their
masters, ultimately unable to defend themselves from abuse; these
characteristics quickly became a reflection of the Chinese people who raised
those horses\footnote{ibid., 305--306; in a conversation between Yang, Chen, and
Zhang as they observed horse mating fights}. When they returned to the
grasslands thirty years later, they found great disappointment in the fenced
pastures. ``In the past, who would dare build a fence on a grassland famous for
its Ujimchinn warhorses?'' asked Chen\footnote{ibid., 510}. The irony was that
they once believed that the spirit of the Mongol warhorse could overcome any
attempts of domestication into a sedentary, labor-based lifestyle, but the wild
horses of their youth had been wiped out by modernization.

Though a work of fiction cannot be relied upon as a statement of historical
accuracy, observable cultural damages and practical ecological disruption did
result from the act of decollectivizing the northern grasslands. These were
highlighted by some of the ramifications of enclosure policies on the
grasslands; instead of keeping livestock confined to specific grazing areas,
herders used enclosures to preserve the pastures assigned to them and allowed
their livestock to overgraze unenclosed areas into severe
erosion\footnote{Williams, 680--681; noting the expansion of fenced land versus
large herds overgrazing unfenced land}. Additionally, the power imbalance
created by introducing subsidy incentives allowed some families to prosper and
others to fail, and generated conflict and disputes over property boundaries
between groups that would otherwise have lived in peaceful
cooperation\footnote{ibid., 684; notes on violent incidences regarding fences}.
These issues were compounded by the drastically different cultures attempting to
survive on the same piece of land, as their incompatible ideologies made it
difficult for them to reconcile their respective needs.

Despite the numerous problems caused by Chinese policy decisions regarding the
use of the northern grasslands, it is difficult to see whether or not there were
other options to address the needs of the Han society. The Chinese could not
tolerate a grey area that was neither clearly Mongol nor Han, thus Inner
Mongolia had to be brought into the Chinese status quo in order to satisfy the
Han cultural preference for clearly delineated spaces. Although the landscape
itself was a shifting environment, because it was a Chinese territory, the
people there needed to function with the Chinese way of life, which meant highly
organized agricultural practices, fenced plots, and measurable work.

Ultimately, the practical needs of a growing population required more land
dedicated to producing food and resources; the steppes were seen as an area that
could provide for the Chinese, if only it could be tamed and organized. The
proximity of Beijing to the steppes region meant that the central government was
very much affected by the ecological changes of the Gobi Desert; continual
desertifcation and destruction of the northern farmlands presented noticeable
problems to the Chinese, from raging sandstorms that buffeted the capital city
to wide-ranging droughts that starved their crops. The preservation of as much
farmland as possible was crucial to their survival. This necessitated
agricultural practices that could be measured, studied, and adjusted according
to a strict scientific rigor. Things could not be studied unless they were held
in place and monitored, thus the enclosure policy presented an implementable
solution to move the region into a more modernized status.
