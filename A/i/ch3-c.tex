\section{The Assimilation Question}

\textit{23 April 2014}

Minority groups generally live under the rule of a state run by a central
majority group; because of this, they face the decision to either maintain their
separation in order to preserve their cultural distinctions, or assimilate into
the dominant culture in order to access status and privileges that would
otherwise be difficult to obtain. As an examination of these issues, we can
follow the trajectories of three well-documented minorities: the Hui, the Amish,
and the Manchu. These three groups share the traits of self-identification as a
unique group with a common history, show visible cultural markers, and are
recognized by their respective ruling states as distinct community with shared
values and interests.

In modern times, all three live in close proximity to the majority center,
inhabiting areas that, if not explicitly mixed with the majority, are at least
not separated by significant geographic features. All three also engage
economically and politically with the ruling state, with representation in
occupations that allow them to trade and interact with outsiders, as well as
participating in government and bureaucratic concerns.  This has provided them
ample opportunity to assimilate, yet they still exist as separate groups.

Dru Gladney refers to the Hui as a nationality created during the census
projects of the late 1900s, combining the historical knowledge of foreign
Muslims who traveled to China as merchants during the Tang dynasty with the
establishing of a nationality-based hierarchy during the Yuan dynasty by the
Mongols\footnote{Gladney, 17--18}. The modern notion of the Hui shows both a
sense of this foreign origin and adherence to traditional Islamic living
practices; while some communities align more with their Arabic ancestors and
others are more devout in their religious practice, the ethnonym of \textit{Hui}
nonetheless calls on an identity that the Hui are capable of identifying with.
In short, even though explicit definition of the Hui minority group in terms of
common traits and cultulral markers proves problematic, no one doubts that the
group exists.

A curious feature of the Hui, in contrast with most other ethnic minorities in
China, is that their population is noticeably dispersed across the country,
occupying more areas and forming a larger percentage of the minority population
in those areas than other groups\footnote{ibid., 27--28}. In addition to their
geographical distribution, the Hui do not have their own
language\index{language} or dialect,
and instead adopt the linguistic traits of their neighbors. Gladney mentions a
specific group of Hui he interviewed who were ``culturally indistinguishable
from the minority group with whom they live, but they identify themselves as Hui
and are recognized by the state as members of the Hui
nationality''\footnote{ibid., 33}. While some Hui wear head coverings that
indicate their religious beliefs, many Hui do not, and instead dress as their
neighbors do. The Hui, then, somehow manage to maintain their status as a
distinct group, regardless of circumstances that make it difficult for an
outsider to determine what makes them different.

One of the main things that sets the Hui apart from non-Hui is their notion of
\textit{qingzhen}, an ethnoreligious concept of purity and truth that is
difficult to identify outside of daily practical living. In some areas,
\textit{qingzhen} manifests as an abstention from pork products, attending
Islamic prayer services, recognizing ancestral lineage from historically
documented Muslims, and maintaining traditional Islamic dress. Not all of the
Hui adhere to all of these characteristics, but recognizing \textit{qingzhen},
regardless of its actual practice, is a cultural marker of the Hui that both
gives the group a self-identification and allows outsiders to see what makes the
Hui different in terms of the social ideals of the group. Additionally, by
giving themselves this ideological definition, the Hui make it easy for then to
refer to things that are not Hui---namely, things that are not \textit{qingzhen}
could not be Hui. Adhering to \textit{qingzhen} gives the Hui a tangible
practice to preserve, which they maintain through their marriage habits, living
spaces, and dietary practices.

Similar to the Hui in this regard, the Amish live according to an
\textit{Ordnung} that defines acceptable living practices that maintain a
limited amount of contact with things that are worldly and
sinful\footnote{Hostetler, 82--83}. They maintain a lifestyle that separates
them, ideologically and physically, from that which is not what they are, and
thus hold a very clear boundary between themselves and the other. In practical
terms, this can be seen in an avoidance of electricity and other modern
conveniences, a moderation of behavior and emotion, plain dress and hair as
regulated by one's church, and a commitment to maintaining the
community\footnote{ibid., 84}.  The Amish are easily identified in modern
American society by observing those cultural markers.

Although the Amish strive to hold themselves apart from things they consider
worldly, they accept that it is both impractical and unnecessary to live in a
completely isolated state. Different levels of interaction apply for different
levels of worldliness; for example, ``interaction with other Amish, with
Mennonite and related groups, or with other religious groups is different from
interaction with the world''\footnote{ibid., 111}. As a necessary part of their
survival, they partake in trade with the outside world for groceries such as
sugar, salt, and flour, and sell their excess production of grain and hay on the
market\footnote{ibid., 113}. Additionally, as they do not own automotive forms
of transportation, they rely on trains and buses for travel over distances that
they cannot practically cover with their buggies. Specific limitations vary
between invididual churches and communities, but in general, their
\textit{Ordnung} dictates acceptable interactions as is needed for their
survival. As a result, they inevitably come in contact with the outside world in
social and economic contexts.

From those interactions, as well as existing within a larger state as a small
minority group, the Amish have encountered problems of maintaining their
\textit{Ordnung}. Hostetler describes them as ``suspended between competing
value systems, subjected to enticements from the external world...confronted
with special problems of coping with regulations and
bureaucracies''\footnote{ibid., 255}. It becomes impossible for the Amish to
ignore the social and technological changes of the world beyond their community,
which induces them to examine their own lifestyles; since they cannot possibly
sustain their separation from worldliness while still holding on to their Amish
identity, they have no way to assimilate without ultimately losing their value
system.

In addition to the temptations of modern living, the Amish have to contend with
a legal system that is not always well-suited to their beliefs. Regulations from
the American government most significantly affect their educational practices;
the Amish generally do not school their children beyond eighth grade, as after
that age, children are expected to remain with their families and develop good
relationships with work and familial responsibilities\footnote{ibid., 188}.
Pennsylvania law, on the other hand, required school up to age seventeen, with
an exception for children from farming families that lowered the age to fifteen;
frequently, Amish were barred from this exception until the formation of an
Amish vocational school that allowed a compromise\footnote{ibid., 262--263}.

Despite the problems the Amish face by maintaining their separate way of living
as minorities, the overall Amish community continues to flourish with an
increasing population growth owed to a high birth rate and low attrition.  The
ideals of their community give them a distinct living space that is incompatible
with outsiders, and they preserve this community by holding a high amount of
value to the very existence of their community. Amish do not marry out to
non-Amish\footnote{ibid., 145}, provide their own education to their children,
and refrain from using technology or services that are antithetical to their
belief system.

At a glance through the history of the Manchus, there exist similar features in
their status as a minority group. The Manchus maintained an explicitly
segregated living space, as well as a roster of all members of their group
within the established banner system; like the Hui and the Amish, the Manchus
knew who was part of them and who was the other. However, their existence as a
group depended mostly on their assertion of their group, without any larger
ideological beliefs supporting their segretation; they used the Manchu name as
``a highly poliicized ethnic label'' that referred to the notion of the Manchu,
rather than identifying distinct lineages or cultural markers\footnote{Elliott,
353}. The existence of the Manchu label implied almost exclusively social
distinctions, such as their status as a ruling class and membership within the
banner system.

It can then be argued that the Manchu identity was far more of a political
contruct than that of the Hui and the Amish, and for that reason, the modern
view of the Manchu makes them harder to distinguish as a noticeable minority
group outside of that official designation. The main aspect of the identity that
set them apart was the banner system itself, which carried the Manchu intoo the
Qing dynasty while the militaristic and linguistic aspects of their lifestyle
faded\footnote{ibid., 354}. The downside to holding the institution itself as
central to a one's group identity, though, is that once the institution was
removed from the collape of the banner system, the main aspect of Manchuness
went with it. Following the Qing dynasty, the Manchus had difficulty maintaining
a strong sense of identity, once they became too far removed from the shared
history of the banner institution\footnote{Rigger, 211}.

In summary, the Manchus were less successful at maintaining their status as a
separate minority group when compared with the  Hui and the Amish. The religious
purity and dedication to an ideal that transcended a material existence gave the
Amish and the Hui a greater sense of importance to their separateness; it wasn't
just that they needed to preserve their community values, but that they
\textit{had} values to preserve at all that set them apart from the Manchus. The
Hui encountered social friction for maintaining their pork abstention amidst a
society favored pork products as a primary protein source, and the Amish face
constant temptation to modern conveniences and political pressure to conform,
but both groups place their committment to their identity at a higher value than
accepting assimilation. For the Manchu, their acculturation seems to lead to an
inevitable assimilation; within the Chinese ethnic minority designation, they
still exist as a valid ethnic group, but as the distance from the period of the
banner system grows, those distinctions become more of a formality than a
recognition of identifiable cultural differences.
