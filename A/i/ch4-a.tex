\section{The Bootstrapping Factor}
\textit{8 October 2014}

Max Weber builds a model of the particular flavor of American
capitalism that depends on a baked-in sense of duty, that all should
adhere to a divine order for everyone to fulfill his potential and duty
towards an economically-viable calling. I argue that he's missing a key
point in his analysis: a defining feature of the workaholic drive that separates
Americans from other modern capitalist cultures is the feeling of necessity that
derives from building something out of less than nothing. This extends to the
belief that failure is unacceptable, a lack of progress is tantamount to
failure, and, in extreme cases, that failure can mean death. It is not only
economics that motivates Americans to work harder than they are expected to, but
a culture that develops out of being forced to pull themselves up by their own
bootstraps, or die.

If we remove money from the equation, Weber's framework collapses. When
he credits the ``devotion to a `calling' of moneymaking''
\footnote{Weber, p. 32} as a major factor in the successes of notable
American cultural heroes, he neglects to understand that the
moneymaking was only incidental to the way in which they conducted
their lives. It happened to be that Benjamin Franklin, Andrew Carnegie,
and others were \textit{highly talented} at generating massive
quantities of wealth, but their deeper motivations came from a driving
need for self-improvement.

A better example illustrating the bootstrapping factor comes from
examining Frederick Douglass's trajectory throughout his life. It
wasn't until later in his life that money even entered into his
calculus; prior to that point, he pursued a desire for self-improvement
and knowledge acquisition before he even understood what implications
and opportunities that would give him. He would not have realized the
power of earning his own wages \footnote{Douglass, \textit{Narrative}
p. 95--97} if he had never realized the possibility of being a free
man. He would not have realized the possibility of being a free man if
he had not learned to read \footnote{Douglass, \textit{Narrative} p.
63--64}. He would not have learned to read if he did not get the sense
that it would somehow \textit{improve} his status as a person
\footnote{Douglass, \textit{Narrative} p. 60}. Because the events
happened in such a sequence, it could not possibly have been an
economic motivation that drove Douglass to better himself.

As Douglass develops a further self-awareness of his own peculiarity,
he recognizes that it was in fact his encounter with adversity that
provided him the circumstances that made him more than what he was born
into. He defines similar `self-made men' as ones who ``are obliged to
come up...in open and derisive defiance of all the efforts of society
and the tendency of circumstances to repress, retard, and keep them
down.'' \footnote{Douglass, ``Self-Made Men'' p. 550} Money had nothing
to do with it; Douglass \textit{had} to raise himself out of his
circumstances because it was what he needed to do in order to survive.
Progress, for him, meant life, while failure to progress would have
killed him. In his case, it was quite literal; once his self-education
progressed to a point at which he could no longer accept slavery as an
option, he decided that he would rather risk the punishment of death as a result of striking back
against a white man than allow himself the inhumanity of being whipped
as a slave \footnote{Douglass, \textit{Narrative} p. 79}.

Douglass doesn't credit only adverse circumstances for his success,
though; he knows that his status as a black man and former slave means
he had much more to work against \footnote{Douglass, ``Self-Made Men''
p. 557}. For that matter, he identifies that great men are ones who
work without regard to anything other than work \footnote{Douglass,
``Self-Made Men'' p. 556, 560}, and the best way to incite work in an
individual is to put him in a situation that seemingly prevents him
from getting what he wants \footnote{Douglass, ``Self-Made Men'' p.
558}. When he tells us that we should measure a man's success ``not by
the heights others have obtained, but from the depths from which he has
come'' \footnote{Douglass, ``Self-Made Men'' p. 557}, he wants us to
acknowledge that the lower someone starts, the greater their path of
accomplishment can be; it is not enough for a man to live with what
life has given him, but that the \textit{process} of achievement shows
more about his character.

You could say that one of Weber's points is that the money itself is
not the sole motivating factor for the spirit of modern capitalism
\footnote{Weber, 20--21}, and thus I am deliberately skewing his
argument, but his framework deals primarily with the sense that there
is something about the economic pressure of capitalism as a strictly
unavoidable part of one's environment \footnote{Weber, p. 18} that
fostered the mindset we are trying to understand. I claim that he has
oversimplified the problem; the driving force behind the spirit of
capitalism is the need to create something out of almost nothing, and
money just happens to exist as a metric by which we can measure the
amount that one has succeeded. Douglass's life, while an extreme
example of the bootstrapping factor, reflects a larger pattern of early
American culture; in order to survive and prove themselves capable of
existing as a legitimate nation, they had to overcome a lack of
infrastructure and established roots that their European spectators had
long solved.


