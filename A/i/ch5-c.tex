\section{The Fear-Cult of Mao}

\textit{16 April 2015}

The disastrous conditions in China following the Great Leap Forward threatened
Chairman Mao's position as a leader of a revolutionary state. Sensing a
potential for loss of faith in his authority, Mao needed a way to secure his
status; the uncertainty the rest of the country had in China's future was turned
into a culture of paranoia through Mao's exaggeration of his own fear. He was
pre-emptively trying to root out potential betrayals within his own officials
before this happened, which meant that it no longer sufficed for his policies to
address actions; rather, his tactic was to create categories and terminology
that meant people could have incorrect thoughts and incorrect personalities,
going beyond just someone's personal history. By externalizing his own mistrust
onto the rest of the country, he created an environment in which he could never
be wrong, while making it impossible for anyone else to be right.

Many of the techniques used during the Maoist campaign were extensions of
procedures already in place prior to the Cultural revolution. The basic format
was that one would be identified as having performed some bad action, required
to make a self-criticism, and then accepting punishments in order to correct
that behavior. As part of Liberation, this process was crucial for the Communist
Party to gain control over an entrenched old class system. This was such an
effective means of establishing a status quo that it was an obvious tactic to
continue for Mao's personal authority campaign.

One of the main benefits of the self-criticism process was that party members
themselves were already expected to use it; in her family biography, Jung Chang
mentions multiple times when her father and mother, both high-ranking Party
members, had to perform a self-criticism in order to address actions they had
taken that were taken to be not purely beneficial to the revolution. Through
leaning on this familiar process, Mao's campaign of fear easily percolated
through the party; self-criticism was always the right thing to do whenever
there was a problem, and thus if Mao wanted you to self-criticize something, it
was surely something that needed to be done. Similarly, if someone did not want
to self-criticize, they would be in the wrong.

Mao targeted people he claimed were taking the ``capitalist road'', which
suggested that they were people who were not yet capitalists, but seemed to be
heading in that direction. By defining his enemies in such a vague, transitional
state, that label had the flexibility to cover virtually anyone it was applied
to, except for Mao himself. Esherick interprets this as having ``implied that
the accused wished to reverse the achievements of China's socialist
revolution''\footnote{Esherick, 283}; in other words, the capitalist road led
away from Mao, and anyone who walked down that path was a traitor to the
Communist Party. The problem, though, is that there isn't a literal road, and
that traitors are not so easy to identify; Mao had to criminalize the act of
considering treason if he wanted to prevent it from happening.

The critical factor in this is that Mao did not trust anyone, so he needed to
create an environment when other people didn't trust each other, either. Chang
claimed that Mao wasn't even sure what a ``capitalist roader''
was\footnote{ibid., 276}, but that he knew they existed and needed to be wiped
out. His declaration of war against an undefinable category of people resulted
in his loyalists scrambling to find those enemies, which meant that they were
practically inventing definitions on the fly. This was a process of challenging
previous structures of leadership, a constant questioning of authority that was
well-suited to the teenagers and young people who were caught up in the
excitement of having a revolution. The Red Guard formed as an extension of Mao's
fear of enemies within the party, amplified by the common adolescent mistrust of
adults.

The categories that the Red Guard divided people into reflected ideas that were
already a part of the revolutionary culture; people were ``red'' if they were
revolutionaries, ``black'' if they were counter-revolutionary, and ``grey'' if
they were ambiguous\footnote{Chang, 294}. While these categories started out
from specific family backgrounds and occupations, they rapidly drifted into more
vague definitions; anyone could be called ``black'' if they were no longer in
the favor of the ``reds''. Furthermore, Mao's insistence that a certain
percentage of counter-revolutionaries existed meant that the ``reds'' felt the
pressure to fill a particular quota, which motivated them to label people as
``blacks'' just for the sake of fulfilling Mao's claim. The vagueness of the
red/black/grey categories meant that people could be classified for reasons from
personal vendettas to purely random whims.

As the Red Guard considered themselves an extension of Mao's will, they believed
that Mao could not possibly be wrong, and thus they could not possibly be wrong,
either. Slogans cited their support of Mao's absolute authority, and they
accepted even his apparent contradictions in ideals. However, this mindless
mass-adherence to one man created opportunities for disillusionment; even if it
was impossible to resist the tide of red, individuals were able to see through
the cracks and question whether or not this was the right thing to do. Chang
tells of times she did not wish to participate in the destruction the Red Guards
expected her to join, even though she signed up willingly\footnote{ibid., 293,
307}. Esherick tells of Binbin, who joined a youth work team when her father was
under Red Guard detention, and quickly learned to undermine the system by
deliberately breaking rules and then self-criticizing for a lenient
punishment\footnote{ibid., 290--291}.

A culture based on paranoia can only go so far; while an effective short-term
means of upsetting the status quo, the inevitable effects of the Cultural
Revolution was breeding discontent and even stronger mistrust of Maoist policy
in the long run. Mao's tactic of making vague statements about broad categories
of enemies with no possibility of clear distinctions meant that no one could
truly understand what he wanted; this ultimately ended up damaging his
credibility and creating a chaotic environment of uncertainty.
