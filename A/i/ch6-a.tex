\section{Nature Versus Nature Versus Nature}

\textit{28 February 2015}

At the heart of the Chinese language\index{language} is a distinction between binaries, with
terminology that grows out of phrases that delineate the interior from the
exterior. It's with this linguistic structure that relationships between human
and non-human entities are formed; things are either outside or inside, land is
either cultivated or wild, the world is either within the confines of
civilization or beyond it. We can further use this structure to consider some of
the complexities involved with environmental conservation in China, as it speaks
to the tension between different ideals and desires regarding the use of the
landscape and its resources. More specifically, this examination is concerned
with mountains, rivers, and forests---areas that are not easily made into dense
human habitation, but provide resources to support a modern society. Without
taking politics too deeply into account, option for such areas include
cultivation for resource and energy extraction, development into designated
tourist sites, or setting aside as wilderness reserves.

A brief overview of the past century of Chinese industrial and economic
development shows that rapid land cultivation and resource consumption was the
main trend; the driving factor of modernization was increased involvement with
the more powerful Western world. Given the state of warfare and economic threats
to the Chinese state, it seemed natural to focus on rapid development in order
to protect Chinese interests, even though in retrospect it seems obvious to see
the environmental damages caused by such short-sighted development. While heavy
land cultivation continues today, that same involvement with the West introduced
additional possibilities for use of less-inhabitable areas.

It's only in more recent decades that places with high resource value were also
seen as tourism\index{tourism} potential. Tourists depend on several
simultaneous factors, such as a middle class with disposable income, private
transportation, and a working culture with vacation and leisure time.
Additionally, tourists need somewhere to go, and developing some of the untamed
wilderness areas in the periphery seemed like a natural thing to do. As observed
by Pal Nyiri, tourism as understood in the Western sense was mostly absent in
China until after the death of Mao\footnote{Nyiri, 3}, when it became more
prevalent as a way to modernize the country and generate revenue. With a large,
growing middle class, China's potential for domestic tourism provides another
option for economic competition on the global market.

The combination of the environmental consequences of resource extraction and
unfettered tourism creates a heavy demand on those areas, which are frequently
places of fragile ecological balance that is easily disrupted. Thus, the third
option is introduced: dedicate certain areas as natural reserves, where human
interaction is forbidden, in order to maintain its wild status. Robert Weller
points to the Western ideals about nature and the need to protect it as further
influences in shaping China's approach to the wilderness\footnote{Weller, 7}.

However, there's a tension between the conservation of nature as a pure object
separate from civilization, and a view of the wilderness as something that does
not necessarily exclude humanity as a component. People who live in non-urban
areas are more likely to act as stewards of the land; we see this in Shapiro's
view of the environmental devastation following forcible relocation of labor
units during the Great Leap Forward\footnote{Shapiro, 140--141}. Ironically,
these are often the people who are moved from the area in mandated conservation
acts that seek to scrub all evidence of human interaction from the area they
intent to protect.

Despite this act of separating humans from `nature'---or perhaps because of
it---there exists a desire in people to interact with this world. Weller
provides a study of tourist responses to nature parks in China and Taiwan, where
people want to touch the rocks, play in the water, and breathe the air; they ask
questions about what can hurt them and what they can use\footnote{Weller,
``Stories of Stone'', 64--104}. In short, there exists some attracting force
between humans and non-human environments. It this simply a curiosity to
experience something out of the ordinary?

Perhaps not. While there is no tradition in China for modern tourism based on
the commodification of scenery and constructed playgrounds, there is a tradition
for regarding the environment as a force of great mystery and fascination,
deserving of both fear and respect. ``Can you eat it? Can you use it for
medicine?'' The encounters Weller describes from frustrated tour
guides\footnote{ibid., 91} come from the naive, but they speak fairly directly
to this fascination with the environment. Those tourists are curious about this
thing that they have been separated from, this 'outside' world that is beyond
their current experiences but they've come to understand as something with
value. It is a driving force behind all living beings to survive, and those
questions are the basis for how humans figure out survival.

Thus, nature is tamed so that people can survive, but in the process, they
become separated from it, and forget what it meant to live in a world where
there was no word that effectively separated humanity from the environment.  The
creation of dedicated natural spaces, while acting on a local need to preserve
the land, acts to further the gap between civilization and the wild.  Since the
act of civilization divorces people from the environment, it also creates a
desire to fill that gap, as modern people otherwise lack the opportunity to
exercise the more primal human drive to survive in a non-human world. However,
the more weight given to creating a fulfilling park experience, the more
emphasis is placed on the uncomfortable realization of self-domestication.

Instead of attempting to resolve this tension entirely, it's perhaps possible to
utilize it as a benefit. Ralf Buckley presses on the need for tourism-supported
conservation efforts, using tourist-generated revenue and visitor management to
reduce impact to threatened ecosystems\footnote{Buckley, 9}. He makes a further
point of leveraging the consumer base of the tourism industry as a way to
support conservation; the revenue generated from tourism is, in some cases,
comparable to what would come from resource gathering, and tourism itself is
dependent on the preservation of desirable areas.

Weller presents an argument for the complicated issue of translating the word
`nature' into Chinese; the modern accepted term \textit{ziran} does not
necessarily carry the English connotations of wilderness, outdoorsy
areas\footnote{Weller, 20--22}. If one specifies `nature' as something that is
`outside' (\textit{waimian}), it calls to mind the separation between interior
and exterior; `nature', then, seems to separate humans from the rest of
existence, which precludes the notion that humans are in any way natural. In
that sense, it seems that conservation is truly impossible, as even the act of
creating a wilderness reserve is an indication of human intervention. This is
obviously an unsatisfactory conclusion, but the complications of this system
means that it is troublesome to maintain discourse about the issues at hand.
