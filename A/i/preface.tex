I think of my post-bacc phase as this: one day, I realized I missed writing
papers, and regretted not engaging with paper-writing in a more meaningful way
during my undergraduate years. I graduated from the English department! How did
I fail to feel satisfied with my paper-writing? (the answer probably lies buried
under the stale beer-sweat of post-adolescent angst)

Here is an archive of my rampage through Carnegie Mellon University's history
department, courtsey of my staff tuition benefits. I'm writing this now as part
of my transition into what I think will become my pre-doc phase: the period in
which I have not yet applied to a graduate program, but am taking much more
serious steps in that direction. Also, now I'll have to pay tax on my tuition
benefits.

It always felt a little unfair to the other undergraduates when I was doing my
post-bacc coursework; not only had I already finished a bachelor's degree, but I
finished it from the same university as they were attending. I was hitting far
below my weight class, and I knew it. What I'm looking forward to as a pre-doc
is that my classmates will have achieved things I haven't yet reached: they've
taken their GREs, they've applied to a graduate program, and, more to the point,
are actual doctoral candidates. Me? I'm just a non-degree staffer filling up my
free time with busywork.

These are presented in chronological order, with one chapter per course;
referrenced sources appear at the end of every chapter. I've made a few minor
typological and typographical edits for the sake of consistency, but other than
that, these essays remain the same as what I turned in to my professors.

\textit{Major Topics:}
    \begin{itemize}
    \item American Cultural History
    \item Chinese Power Structures
    \item Aesthetics/Visual Language
    \end{itemize}

\textit{Themes:}
    \begin{itemize}
    \item Language, and how it shapes perception
    \item Delineations/transitions
    \item Informations transfers
    \end{itemize}
