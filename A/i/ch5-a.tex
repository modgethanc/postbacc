\section{Success and Failure in the Qing Exams}

\textit{2 February 2015}

In the latter half of the 19th century, China's Qing dynasty relied on an
imperial system of rule inherited from a long history of adherance to an
examination system that provided the structural framework on which the issues of
morality and effective government were hung. While a crucial tool in training
government officials, the exams also enforced a hegemony that favored strict
adherence to the structure itself, which often sacrificed the value of actual
moral teachings in the process. The result is that people who came from a long
line of successful officials were more likely to continue experiencing success,
while new individuals attempting to enter the field were frequently left behind.

We can see this split illustrated clearly by studying two separate paths of
Chinese officials, Ye Kunhou and Liu Dapeng. Both men were devoted to the
Confucian ideals they studied in preparation for their exams, and both men
seemed dedicated to serve China and the people under their jurisdiction. They
initially struggled with obtaining degrees, failing many exam cycles while their
peers succeeded. However, they differed in familial history, and their
subsequent responses to the imperial examination system differed in profound
ways.

The success rate for degree candidates was extremely low; it's estimated by
historian Joseph Esherick that 1--2\% of candidates pass the first level, 2--3\%
of those pass the second, and fewer than 10\% after that pass the third level
\footnote{Esherick, 19--20}. It was standard that candidates would struggle, and
common that they would fail. But despite the overall rarity of degree holders,
the Ye family recorded a large number of degree holders within their members
\footnote{ibid., 27, 30}; this suggests that the success of previous ancestors
contributes to the success of future descendants.

Ye Kunhou had the good fortune of belonging to a lineage with strict family
regulations and knowledge passed down from his ancestors; those guides and
support maintained the long history of status experienced by his family members.
Esherick in fact credits this longevity of the Ye lineage with some of the later
members' success; he specifically cites the lineage school formed more than ten
generations before Kunhou's time as a crucial factor in the education of Ye
descendants\footnote{ibid., 26--27}. Without the success that came from Ye Hua,
who established the school, the Ye family may not have seen as many
degree-holding members. Indeed, during the period of time following the collapse
of the Ming dynasty, the family suffered from losses that included the lineage
school, and a decline in degree-holders\footnote{ibid., 27--28}.

Even though Kunhou entered the examination scene after a period of low status
for the Ye family, he received crucial support in his early childhood through
attending the family's school and receiving strict care and encouragement from
his mother\footnote{ibid., 37--38}. Though it took him twenty-five years to pass
the exams, he never seemed to lose sight of the bigger picture; during his
career, he was both a devoted son and a caring official who upheld strong
Confucian ideals, and he passed his success down to his descendants.

On the other side of the issue, Liu Dapeng came into the examination system with
no history of family success. His father sent him to school to learn basic
skills needed for business\footnote{Harrison, 25}, but he showed enough promise
in early performance that he was allowed to continue his studies\footnote{ibid.,
26}. In her biographical history of Liu Dapeng, Henrietta Harrison describes the
education system as both ``a form of moral indoctrination''\footnote{ibid., 25}
and ``formulated by the government for examination purposes''\footnote{ibid.,
26}; this is precisely the split between the philosophical meaning of the
classical texts and the practical realities created by the exam system that made
it difficult for men like Liu Dapeng to succeed.

Liu Dapeng progressed through his education by taking the moral lessons to
heart, and accepting the literal commands for filial piety and serving the
Confucian ideal. His commitment to this ``moral indoctrination'' prevented him
from performing well on the exams, as he slowly came to the conclusion that most
of his classmates pursued educational success for the sake of gaining
recognition, while he failed them year after year even as he endeavored to live
up to the standards he was taught\footnote{ibid., 37}.

After encountering corrupt degree-holders, Dapeng lost faith in the system that
taught people only to memorize the syntax of the classics while neglecting to
learn the actual lessons presented\footnote{ibid., 38}; it was only in
redefining the moral code for himself that he was able to pass them at all.
However, his anxieties about being the perfect scholar and the flawless son
followed him through the rest of his life. As his sons progressed through their
education, he pressured them to pass the exams he resented in his youth, but
both of his sons ended up as failures\footnote{ibid., 76--77}.

Liu Dapeng's criticism of the imperial exam system as a whole points back to the
original purpose of the degrees; earning degrees brought power to the
individual, which translated to prestige for the family. The exams, then, were
ultimately a system that rewarded obedience to the system itself, and
reinforcing a culture of hegemony. For people like Ye Kunhou, a family history
of succeeding within the system encouraged them to continue performing well,
while that same aspect alienated men like Liu Dapeng by condemning them to
failure.
