\section{The Immigrant Ethic and the Spirit of Escapism}

\textit{14 December 2014}


At the core of Americanism is a dream, described by James Truslow Adams as the
belief that all people have the capacity for self-fulfillment \footnote{Adams,
404}. This is an ideal built out of the demands a culture places on itself when
one of the parameters of its founding is to choose to escape from an
unacceptable lifestyle and take on the risks that come from plunging into the
unknown. It is a paradoxical belief that still manages to sustain itself,
because its very structure dictates that to reject it is to reinforce it.  In
short, the American Dream is a self-propagating concept that rewrites the fear
of failure into the desire for success.

In the 1990 Nintendo game \textit{Super Mario World}, each level is a linear maze with an
explicit entrance and exit. Some levels have secondary exits that require the
player to notice an abnormality in the map and exercise some creativity to
exploit it. Towards the end of the game, though, one of the key levels breaks this model; the obvious
exit does not unlock any forward progress, and there are no visual clues to
hint the location of a secondary exit. The only way to move beyond this
level is to fly far off the edge of the screen, out of the visible playing
field, and wait to land on a small island that contains a hidden door.

At age eight, I had little ability to deal with this; I only found the secret
exit because someone who watched me struggle with the level insisted I try
flying off the screen. ``If you die, you die, so what do you have to lose?'' I
was assured, and I followed those instructions because my only other choice
was to fail to beat the game. Later, I understood that this approach was what immigrants, including my
own parents, needed to believe in order to stake everything on their dreams and
start a life in a new world.

Prior to the founding of America, Benjamin Franklin already presented a
success story---arguably, the first great American success story. His
timeline is one that glorifies working ceaselessly, whether or
not he even did the same himself. That narrative of declaring one's starting
circumstances as undesirable and finding the means to escape is one that has
been repeated and romanticized through countless historical figures and
fictional heroes in American lore. It's an obsession that stems from the subtext
presented in Franklin's way of life---it's not just that working hard is the
true path to success, but that it requires a preceding step: first, you must
choose to change your circumstances.

Obviously, not everyone makes that decision, or at least not at the magnitude
that Franklin did. It is simple to come up with an explanation; most people
fear failure, described by Sennott as ``the most uncomfortable phenomenon of
American life'' \footnote{Sennott, 183} in his study of problems imposed by
American class divisions, and most people do not experience
a level of desperation in their lives that is sufficient to demand change.
However, this just emphasizes the effects of those who do decide that the only
way forward is to stake everything on a leap of faith.

Such a drastic decision fuels the momentum of immigrants, who are perhaps a
stronger product of this Dream than those with much longer generations' worth
of history tied to America. Regardless of their originating culture, the
immigrant narrative shares a common thread: failure is an impossibility,
because it is simply not an option to not make it in the new world. You fly off
the edge of the screen, or you die trying. Immigration
is a radical form of escapism, a nuclear decision to up sticks and start over
in a place that categorically \textit{must} be an improvement. With that much at
stake, there is no looking back.

The assertion, then, is that by choosing to escape, one must have already
chosen success. This is clearly not an objective fact---the possibility of
failure is, of course, always there. However, once the stakes become high
enough, it's no longer meaningful to consider outcomes of failure, because
only success matters. Specifically for immigrants, the conditions of success
lay in establishing a secure life outside of their originating country, which
means that failure results in returning to the old country; by deciding to immigrate,
the decision has already been made to never fail. It's a
redirection of the terms of success outlined by the status quo that means as
long as you never return, you've made it. This is the step that counteracts
the anxiety Sennott ties to the American view of failure \footnote{Sennott, 183}; when
failure has been removed from consideration, the only paths in view lead to
success.

As a culture founded on escapism, the value system shifts from outcomes to
actions. If failure is removed from view, success as a concept loses meaning
as well.  Since it's always a possibility to throw everything out and start
over, people must then derive value from something other than the results
produced by work; an unavoidable fact is that there is always work, so why not
base a value system on the work itself, rather than its products? Performing
work is an expression of agency; making choices is an expression of freedom.
What comes afterward is immaterial, because the process of those actions in
and of themselves reinforces their value. In other words, it no longer matters
where you go in life, but how you get there.

Thoreau says of a life, ``meet it and live it; do not shun it and call it
hard names. It is not so bad as you are.'' \footnote{Thoreau, 290} He bases the value
of a life on how it is lived, not what it achieves, a conclusion he draws only by
rejecting the standard success model of the status quo and searching for a
more reliable path of self-fulfilment. This model of escapism is no different from
Franklin's, only it dictates working exactly as much as one should, not to excess.
The immigrant ethic is to speak Franklin, while living Thoreau.

This is the nature of the
American Dream. It is a dream that demands a constant challenging of
circumstances, even if such circumstances were a direct result of
pursuing the dream itself. A doctrine that requires rejection of itself as a
core tenet can only persist through this state of self-denial; a cultural rejection of
unfavorable circumstances similarly self-perpetuates through the ideals formed by
the desire to escape.

