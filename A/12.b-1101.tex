Zeng 1
Vincent Zeng
Sandage
AMRN PTLT HMR MK TWN
November 2, 2012

Feeding the Comedian Ego

Media producers can be split into two categories: ones that appear to inform,
and ones that appear to entertain. Under the guise of providing information,
propaganda and advertising developed for American post-war mass media in a way
that impressed upon the still-infantile state of its consumers that specific
rules and standards must be met. However, this same generation proved to be an
apathetic, distrusting, rebellious one; they soon turned to entertainers as
their primary source for media consumption. Consumers' active choice of
preferring entertainment over information gave comedians of the era a new sense
of importance, which in turn fed an egotism for that spotlight that would not
have existed with any other sort of audience.

Post-war America experienced a rapid proliferation of media which instigated a
shift in the way communication propagated. Between technological innovations
that allowed a more direct transfer of messages from creator to receiver and
economic growth that increased accessibility to information channels, a culture
of widespread, simultaneous media consumption sprang into existence. As the
number of consumers rose, media producers noticed that they wielded a power to
influence a larger percentage of the population than ever before, even as it
gave them greater responsibility of presenting information in a reliable manner.

The growing threat of nuclear war in the 50s created a tense and delicate
environment that called precisely for a more responsible media. In ``Explosive
Issues: Sex, Women, and the Bomb,'' Elaine Tyler May described this as an ``era of
the expert'' (May 155), as well as ``a heightening of the status of the
professional'' (May 156). When Americans needed someone to tell them everything
was going to be okay, along came the Federal Civl Defense Administration; it
gave agency to women by presenting them a sense of power over their lives—so
long as they did as they were shown through programs such as Jean Wood Fuller's
``Home Protection and Safety.'' During that period of uncertainty, this act of
showing people a method of taking back control was intended to ease the
population; if women followed the instructions presented to them by those
experts in the media, ``they would face the danger of an atomic attack without
fear'' (May 161).  With that, Americans were taught to trust the media, that
magazine articles, radio broadcasts, and television programs would provide them
with the information they needed to know in order to protect themselves from an
otherwise mysterious and terrifying threat.

Soon, this purely informative presentation blended with more commercial
endeavors; fear-reduction became a marketable commodity, and ``[c]ontractors
commercialized the idea by creating a variety of styles and sizes [of bomb
shelters] to fit consumer tastes'' (May 162). Those contractors piggybacked on
Fuller's domestic safety programs, worming their way into the category of media
producers that inform the viewer; people who feared nuclear attack could protect
themselves with a well-stocked pantry, good first-aid skills, and the purchase
of a commercially-produced bomb shelter designed by experts. This resulted in
the culture of what Jackson Lears described as being symbolized by a ``pink lamp
shade''; an object that was widely accessible to a significant portion of the
population and provided a ``specter of conformity'' (Lears 44). During the cold
war, such conformity gave some Americans the comfort they needed, but not all of
them.

Dissatisfaction with cultural conformity existed during the time, as evidenced
by the very essay from which Lears lifted that symbol; he agreed that the
``attack on corporate-induced conformity was pointed and well placed'' (Lears 45)
even as he didn't seem to see alternative options presented other than the
forced acquisition of a pink lamp shade. He went on to paint a picture of
post-war America that involved ``theatrical social performance,'' ``the irrelevance
of politics,'' a ``system that fed the body but starved the soul'' (Lears 45). In
other words, he set up a situation in which the general American public was in
desperate need of something to fill the frustrating space left between fear and
an uncomfortable status quo.

Enter the comedians of post-war America: Mort Sahl, Vaughan Meader, Lenny Bruce,
Dick Gregory, and countless others. Characterized by a quickness in wit and
performance, more improvisation, and a distinct self-awareness of the art of
stand-up comedy, this new generation of performers demonstrated an egotism that
had yet to be seen in the field of humor. While educators and advertisers were
leveraging the growing reach of the media to influence the masses—for better or
for profit—comedians took advantage of the wider audience in order to entertain
more people faster. As commercialization of the media piggybacked on educational
resources, entertainment piggybacked on both, benefitting from the heightened
emphasis on the importance of a media-dictated culture. They experienced a quick
rise to success, owing both to the relative ease with which their performances
could be brought to the public through television, radio, and records, and to
the public's willingness to accept political humor as a way of addressing their
dissatisfaction with a hegemonic state.

The introduction of comedians directly into the homes of media consumers meant a
sharp increase in audience size; Peter Robinson noted that they ``were quickly
able to popularize their new and daring brand of political humor in front of a
virtual audience that numbered in the millions'' (Robinson 122). The act of
performing already requires a base level of narcissism; the ability to
simultaneously reach far more people than a physical venue could ever hold gave
rise to a new level of ego-boosting. This dramatic shift in the ratio between
content producer and content consumer changed the character of comedic
performance. Additionally, consumers were given a sort of agency over the media;
innovations in recording technology meant that ``they could conveniently and
cheaply re-create their own music or comedy performance space at home'' (Robinson
121); comedians became thusly idolized and propagated.

Ultimately, it was these comedians who counteracted the hegemony described by
May and Lears; with them, the public found an outlet and voice for their
frustration against an unsympathetic status quo. At the same time, they achieved
an awareness of their power as critics of the system. The combination of a
massive audience that needed and loved them with their self-awareness of
importance fed the growing ego of the comedian, which resulted in the
development of a brand of humor unique to post-war America.

Works Cited
Lears, T.J. Jackson. ``A Matter of Taste: Corporate Cultural Hegemony in a
Mass-Consumption Society''. 1986.
May, Elaine Tyler. ``Explosive Issues: Sex, Women, and the Bomb''. 1988.
Robinson, Peter. The Dance of the Comedians. 2010.
