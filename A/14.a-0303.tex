\documentclass{article}[12pt]
\usepackage{mla}
\begin{document}
\begin{mla}{Vincent}{Zeng}{Professor Donald Sutton}{79-212 China and Its Neighbors}{\today}{A Culture of Disunity}

As a nation with an expansive territory, China's borders contain a necessarily
diverse collection of cultures. This presents obvious difficulties for
creating and maintaining a unified state; the further from China's
geographical and cultural center, the more complicated this issue becomes. The
Uyghurs and the Mongols are very different groups, both in their semantic
distinctions from the Han center and their syntactic differences as ethnic
groups.  However, they have the shared experience of being peripheral groups
that have undergone policies and reforms instituted by the Chinese government
in an attempt to bring them under a unified national identity. How either group
responded to this force of change depended on their respective cultural
backgrounds.

The geography of the Mongol region favored the development of communal,
migratory societies. Harsh steppe landscape meant a delicate ecology,
with grasslands that were prone to damage from overgrazing and erosion. As a
result, the Mongol lifestyle followed an annual migration pattern that allowed
them to raise livestock carefully without causing great disruption to the
land. This necessitated a strong community spirit and awareness of the
overall health of the region, which promoted bonds across the steppes that solidified the Mongol
identity. In spite of their mobile lifestyle, the Mongol social structure
developed as collaborative and homogeneous once they achieved a unified state.

In addition to geography, the symbolism of
Chinggis Khan throughout history contributed to the collective Mongol identity.
Almaz Khan describes this as ``one of the basic identity symbols for both the
Inner and Outer Mongols''\footnote{Khan,
248}. He goes on to trace the early period of the region, when there did
not yet exist a strong sense of Mongol unity, until Chinggis Khan's invocation as a
revered ancestor of the people in the region served as a way to legitimize the
claims of power among different tribes \footnote{Khan, 252}. In more modern
times, Chinggis Khan represents the Mongol people's struggle for status within
the Chinese state; it exists both as a common thread of identity for the
Mongols themselves, and as an object that the central government either revered
or suppressed depending on its relationship with minority cultures
\footnote{Khan, 267}.

Regardless of how Chinggis Khan has been utilized to represent the Mongols,
the important thing to note is that such a symbol exists at all; the
existence of a commonly-accepted cultural hero speaks to the existence of a
commonly-accepted cultural identity. Without that identity, there would be no
hero. The strength of such a symbol feeds into the strength of the ethnic
consciousness, and vice versa. So long as the symbol of Chinggis Khan existed,
the Mongols were able to relate to a sense of unity springing from a single
foundational icon of their history, no matter how the mythology of that icon was
treated.

Regarding the Uyghurs, such a united identity did not
exist prior to the PRC designation of the Uyghur ethnic group. The Xinjiang
region is mainly uninhabitable desert, across which travel is a great hardship.
Settlements flourished around individual oases, which could more easily sustain
life; communication to other oases occurred less frequently the longer
the distance. Social and family groups remained very tightly bound to local
areas, as pride for one's home oasis came from the practical matter of
survival; without a large, geographically close network of connections, one was
unlikely to prosper. Due to the distances between oases, trade and cultural
mixing happened more often across the border into neighboring countries than
within the Xinjiang region \footnote{Rudelson, 40}.

Due to such disparate cultural values and isolated communities, the Uyghurs never
felt a sense of unity with other people called Uyghurs. Even as the
Uyghurs must address their unavoidable status as a people united under the
Chinese designation, they still found difficulty reconciling their disparate
histories, and maintained loyalties to their home oases above their connection to
the idea of a larger ethnic group \footnote{Rudelson, 143}. In addition to the
problems of cross-oases identities, the Uyghurs experienced distinction between
classes that presented even stronger disconnects of cultural values and
self-identity. The intellectuals considered themselves as Turks before
any other group, the merchants as pan-Chinese, and peasants as Muslims
\footnote{Rudelson, 118}. Most notably, no group considered themselves
primarily Uyghur, the term that the Han assigned to all three groups.

With the lack of a homogenous identity, the Uyghurs also have not maintained a
common cultural hero. Rudelson identifies a number of historical figures that
have maintained some status as Uyghur cultural heroes, but none of them managed
to take hold of a universal position as something all the Uyghurs identify
with. The poet Abdukhaliq ``Uyghur'' mostly resonated with intellectual
nationalists\footnote{Rudelson 145}, eleventh-century scholar Mahmud Qasqari
represented thestrong Islamic identity among the Uyghurs\footnote{Rudelson,
153}, the former Xinjiang leader Saypidin Azizi fell out of favor with the
majority of Uyghurs for sympathizing too much with the Han\footnote{Rudelson, 155}. Even if
one particular figure received heavy support from one group during one period,
the disparity between the self-identification of each group resulted in highly
polarized responses across the region.

Thus, as individual ethnic groups, the Mongols and the Uyghurs differ not only
in the specifics of their language and cultural values, but also in how they
fit into the syntax of their designation as ethnic groups. Most notably, the
Mongols present a more homogenous identity, while the Uyghurs span multiple
groups of people who are not yet able to relate to each other under a common
identity. In part, this is owed to the relatively new status of the Uyghurs as
one designated group, as the Mongols were similarly scattered before they were
unified under Chinggis Khan. However, the syntactical differences of the two
groups provided a logistical challenge for the Han, whose unification policies
toward ethnic minorities seemed to assume a universal and homogenous
application.
 
\hrulefill

As part of establishing a system of minority unification, the Chinese
central government developed an educational system to both preserve ethnic
boundaries and enforce national unity. A national standard education in such a
diverse country has two major boundaries to overcome: linguistic variance, and
cultural incompatibility. To address those boundaries, primary education in
areas with a high population of ethnic minorities included language-specific
schools meant to enforce the usage of Mandarin as the standard language for
the Chinese nation\footnote{Borchigud, 282}, and study programs emphasizing the
values and histories of the Han people\footnote{Borchigud, 293--294}. They also allowed special provisions for students with particular dietary and lifestyle requirements, such as separate dining areas and options for language education.

For both the Monols and the Uyghurs, an education system that emphasized
Mandarin meant that in order to succeed, students must learn Mandarin as their
primary language; however, in both cases, this was often at the cost of losing
their own culture. Somewhat ironically, by increasing their connection to the
Han group, it hurt their connection with their own group; Mongol stuents who
attended a special ethnic boarding school ``felt that it lowered their
otherwise equal social status with the Han and other ethnic groups''
\footnote{Borchigud, 286} and Uyghur intellectuals seeking to improve secular
education and modernization ``struggle to preserve their Uyghur culture and
language while maintining and developing the ability to compete on a national
level'' \footnote{Rudelson, 122}.

In addition to the negative effects of pushing the Mandarin language onto
ethnic minority students, the creation of special ethnic schools increased the
awareness of those groups, both within the groups themselves, and from the Han
that encountered them. A mixed Mongolian-Han boarding school saw that
``students of each group formed a separate cultural circle, within which they
gradually built up mutual trust by sharing similar experiences of language,
ethnicity, and economic status'' \footnote{Borchigud, 288}. Rather than
building cross-cultural connections, the mixed schools emphasized the
differences between ethnic groups, effectively alienating one group from
another.

Among the Uyghur intellectuals, a split was created solely due to the
introduction of Mandarin-based education; students were permitted to choose
schools in either Uyghur or Han, resulting in the creation of separate cultures
within the intellectual class depending on which language one used throughout
school. Those who studied in Mandarin were viewed as not true Uyghurs, while
those who studied in Uyghur were considered to have recieved a poorer
education\footnote{Rudellson, 127--129}. This internal division would not have
existed if not for the ethnic eductation system imposed on the Uyghurs; even
though the literacy and success rates for students who studied in the Han
language increased, the conflict introduced into the region as a result bred
distrust and arrogance within the Uyghurs themselves.

Ultimately, though the Mongols and the Uyghurs are noticeably
disparate groups, they shared the experience of the effects of being defined as
ethnic minorites by the Chinese central government. The Han provided them with
a system that rewarded those who favored a pan-Chinese nation, as those willing
to participate in the educational and economical systems of the center would
prosper, while those who resisted found hardships. However, to accept that
system meant accepting an ethnic designation imposed from an outsider; this was
an easier thing for the Mongols, who had a unified identity on which that
designation was based, but still presents a problem to the Uyghurs, who are
only developing their identity as this proccess occurs.

\begin{workscited}
	
	\bibent
	Borchigud, Wurlig. ``The Impact of Urban Ethnic Education on Modern Mongolian Ethnicity, 1949--1966''. \textit{Cultural Encounters on China's Ethnic Frontiers.}. 1995.

	\bibent
	Khan, Almaz. ``Chinggis Khan: From Imperial Ancestor to Ethnic Hero''. \textit{Cultural Encounters on China's Ethnic Frontiers}. 1995.
	
	\bibent
	Rudelson, Justin Jon. \textit{Oasis Identities}. 1997.
\end{workscited}
\end{mla}
\end{document}
